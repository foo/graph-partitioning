\documentclass[conference]{IEEEtran}


\usepackage{cite}
\usepackage{amsmath,amssymb,amsfonts,mathtools,amsthm}
\usepackage[noend]{algpseudocode}
\usepackage{algorithm}
\usepackage{url}
\usepackage{tikz}
\usepackage{comment}


\newtheorem{theorem}{Theorem}
\newtheorem{proposition}{Proposition}
\newtheorem{claim}{Claim}
\newtheorem{lemma}{Lemma}
\newtheorem{fact}{Fact}
\newtheorem{corollary}{Corollary}
\newtheorem{rem}{Remark}
\newtheorem{observation}{Observation}
\newtheorem{definition}{Definition}
\newtheorem{property}{Property}




\DeclarePairedDelimiter\pair{(}{)}
\DeclarePairedDelimiter\set{\{}{\}}

\DeclarePairedDelimiter{\ceil}{\lceil}{\rceil}
\DeclarePairedDelimiter{\floor}{\lfloor}{\rfloor}

\newcommand\mahmoud[1]{\color{green}\textbf{\\ Mahmoud: #1}\\\color{black}}
\newcommand\stefan[1]{\color{blue}\textbf{\\ Stefan: #1}\color{black}}
\newcommand\maciek[1]{\color{brown}\textbf{\\ Maciek: #1}\color{black}}


\begin{document}

\section{ToDos}

\begin{enumerate}
    \item Make sure that we loose nothing while preparing the short paper (backup: in branch, in directory(svn-style branch)?)
    \item Apply the ACM style and see how much space do we have
    \item Decide on the subset of results to publish (I would minimize those: if a figure on LB fits, it would be vital to include it)
    \item (and hence, draw a figure)
    \item Decide on the order of results
    \item Include intro, related work etc. from PhD (Machine Migration, DYnamic Mapping, Related Work, Problem Definition, References)
    \item Couple iterations on existing results (these are mostly fine already, though)
    \item Make an ACM account for the submission and sum up all the steps needed for the submission
\end{enumerate}

\section{Introduction}

Main technical contribution of this paper is $\Omega(k\ell)$ lower bound for any deterministic algorithm for Online Balanced Partition problem.
Previous known lower bound were $\Omega(k)$ (cite Marcin-Chen).
The gap between upper and lower bound is still substantial, as the best known algorithm is $O(k^2\ell^2)$ (cite Marcin-Chen).

We contribute to the search for the optimal algorithm by presenting $O(k\ell)$-competitive algorithm for the special case where the perfect partition of the input graph is assumed and every algorithm must colocate every communicating pair.
Note that our $\Omega(k\ell)$ lower bound uses a perfect partition input graph and colocates communicating pairs, and hence we provide a tight result for such scenario.

\maciek{Future directions of research: How well does the Perfect Partition Learner performs when applied to a single epoch in general online clustering problem (without the perfect partition?)? Epoch = C * l * (k-1) external requests.}

\section{The Model}

Formal introduction of the general model.

 We say that an online algorithm DET is component-respecting if it never reaches a configuration with a component split between clusters.
  %Recall that if DET resides in a configuration with a component split between clusters, we issue intra-component requests until
 If DET splits a component, we issue intra-component requests until DET joins it.
 If it never joins a component, it is not competitive.

 \maciek{+ description of perfect partition + ground components model}
\section{Warm-up: lower-bound $\Omega(\ell)$ for $k=3$}

\maciek{The intention of this section is to present a simpler version of a lower bound (that should be skipped if we are short on space) AND to present a lower bound for the case with augmentation and discuss the gap}

\begin{theorem}
  The competitive ratio of any deterministic algorithm for Online Balanced Partition with $k=3$ is $\Omega(\ell)$.
  \label{th:lb_omega_l}
\end{theorem}

\begin{proof}
  \maciek{ToDo}
\end{proof}

\begin{observation}
  The lower bound of $\Omega(\ell)$ can be generalized for arbitrary $k = c\cdot 3$.
  The adversary starts with a serie of ``glueing requests'', so that each cluster contains 3 components of size $k/3$.
  We repeat the argument of Theorem~\ref{th:lb_omega_l} using these glued components instead of singletons.
  OPT pays $O(k)$ for migration of one such component, and ALG pays $\Omega(k \cdot \ell)$.
  \label{obs:lb_general_k}
\end{observation}

\begin{observation}
  The lower bound of $\Omega(\ell)$ holds even with $1/3-1/k$-augmentation, i.e., in scenarios where the capacity of each cluster of the algorithm is $(1+1/3)\cdot k - 1$, while the capacity of OPT's clusters is $k$.
  To this end we use the construction from Observation~\ref{obs:lb_general_k}.
  We observe that each cluster of the algorithm can fit exactly $3$ components of size $k/3$.
\end{observation}


  \maciek{Note that this does not contradict the existence of $O((1+1/\epsilon)\cdot k \cdot \log(k))$ algorithm of Avin, Bienkowski et al, as the algorithm requires $2+\epsilon$-augmentation.}

  \maciek{Note that we have a gap here. What is the maximum augmentation (between 1+1/3 and 2) that still has the $\Omega(\ell)$ lower bound?}

\section{Lower bound $\Omega(k\cdot \ell)$}

\maciek{I assume $\alpha = 1$ to simplify calculations, but everything should hold for general $\alpha$}

In this section we consider the Online Balanced Partition problem.
We show a lower bound of $\Omega(k \cdot \ell)$ that uses an input that has a perfect partition.
Therefore, this result is a lower bound for the general Online Balanced Partition problem, as well as a lower bound for Online Balanced Partition with perfect partition.
The latter complements an upper bound $O(k \cdot \ell)$ for Online Balanced Partition with perfect partition that we present in Section TODO.

\begin{theorem}
  The competitive ratio of any deterministic algorithm for Online Balanced Partition is $\Omega(k\cdot \ell)$.
\end{theorem}

\begin{proof}


 
  Fix a component-respecting deterministic algorithm DET.
  Fix a cluster $A$ containing nodes $a_1, \ldots, a_k$ and a cluster $B$ containing (among others) a node $b_1$.
  The adversary begins with issuing $k-2$ internal requests $(a_i, a_{i+1})$, so that nodes $a_1, \ldots a_{k-1}$ form a single component.
  These internal requests are free for OPT, and it does not perform additional actions at this point.
  The next request issued by the adversary is an external request $(a_k, b_1)$. %, which at this point is the request $(a_k, b)$.
  Note that DET must relocate the component $\langle a_k, b_1\rangle$ to some other cluster, as otherwise it is not component-respecting.
  OPT also relocates the component $\{ a_k, b_1\}$ to some carefuly chosen cluster $D$, and it evicts nodes $d^{(1)}$ and $d^{(2)}$ (to be fixed later) from $D$ to the previous locations of $a_k$ and $b_1$ (i.e., to the clusters $A$ and $B$).
  For these actions, OPT pays $2$, and DET pays at least $2$.
  We call the requests issued so far the requests $I_1$.


  Now, we construct a sequence of requests $I_2$, but we do not feed them to DET yet.
  Instead, we simulate DET's actions on $I_1I_2$ to decide at which point to terminate the input sequence earlier, i.e., we choose a prefix $I_2'$ of $I_2$ and we feed DET the input $I_1I_2'$.
  %\maciek{We can produce $I_2'$ on-the-fly, but this was convinient to write it down this way}


  The input $I_2$ consists of requests $(x^{(1)}, c(x^{(1)})), (x^{(2)}, c(x^{(2)})),\ldots$.
  The node $x^{(i)}$ is a node in the current configuration of DET that is co-located in one cluster with the component $a_1, \ldots, a_{k-1}$.
  After a request $(x^{(i)}, c(x^{(i)}))$, the node $x^{(i)}$ is a part of a non-singleton component and must be relocated to some other cluster (as its current cluster hosts a component $a_1, \ldots, a_{k-1}$).
  Then, DET places the node $x^{(i+1)}$ in place of $x^{(i)}$, and the next request concerns that node.
  % Initially, $x^{(1)} = a_k$, and the next node $x^{(2)}$ is the that DET locates in place of the
  The sequence continues as long as after serving the request, a perfect partition of the input graph exists.

  For each $x^{(i)}$, we set $c(x^{(i)})$ to some (to-be-determined) node that in the initial configuration was co-located with $x^{(i)}$.
  While producing an input sequence, we would like to additionally order the initial nodes $V^C$ of each cluster (i.e., order the nodes that were present in each cluster in its initial configuration).
  The ordering needs to be produced on-the-fly, as this ordering is used to produce subsequent requests.
  
  %Note that in the initial configuration, nodes of one cluster are indistinguishible for DET \maciek{and hence we can order them on-the-fly}.
  
  Now, we formally define $c(x^{(i)})$, and the ordering of nodes in the initial configuration.
  Consider a set of nodes from an initial configuration of a cluster $C$ (where $C \neq A$).
  By $v^C_1$ we label the first node of a cluster $C$ that appears in the sequence $\langle x^{(i)}\rangle$.
  At this point, we fix an arbitrary other node of $C$ and we label it $v^C_0$, and we set $c(v^C_i) = v^C_0$ for all $i$.
  For the following $k-2$ nodes of the cluster $C$, we label them $v^C_2, v^C_3, \ldots$ in the order of appearance in the sequence $\langle x^{(i)}\rangle$.
  %For a node $v^C_p = x^{(j)}$, we set $y^{(j)} = v^C_0$.
  If some nodes of $C$ never appear in the sequence $\langle x^{(i)}\rangle$ %(e.g. DET might terminate prematurely or might reside in a configuration with components split)
  , we number them arbitrarily, with the restriction that the nodes that do appear in the sequence have smaller number than the ones that do not \maciek{This should not happen at all for component-respecting DET, though}
  
  Now, we claim that given the input $I_1I_2$, every component-respecting DET 
  % (that does not terminate prematurely and never resides in a configuration with components split)
  must produce a sequence $\langle x^{(i)}\rangle$ of length $L = k\cdot \ell - (k-1) - 2 - (\ell-1) - 1$.
  Consider a request $(x^{(i)}, c(x^{(i)}))$.
  If in this configuration at least one singleton component is present (other than the components of $(x^{(i)}, c(x^{(i)}))$), then serving this request is possible (i.e., the perfect partition exists), and hence a singleton node appears as $x^{(i+1)}$ and the sequence continues.
  Otherwise, no perfect partition exists, as we have a component $\{a_1, \ldots, a_{k-1}\}$ of size $k-1$.

  Hence, the length of a sequence $\langle x^{(i)}\rangle$ is closely related to the number of singleton components.
  Before the arrival of the first request from $I_2$, the number of singleton components is $L + (\ell-1) + 1$, and after serving the last request, the number of singleton components is $1$.
  %Every node $x^{(i)}$ must be a singleton component, otherwise DET resides in a configuration with components split (and such DETs either leave such configuration eventually or issue an infinite cost).
  After each request, we join this singleton component with another component.
  This join decreases the number of singleton components by either $1$ or $2$: if we join two singletons $\{v^C_0\}, \{v^C_1\}$ (there are $\ell-1$ such joins: for all clusters but $A$), then the number of singletons decrease by $2$, and it decreases by $1$ otherwise.
  We conclude that the number of requests $\langle x^{(i)}\rangle$ to decrease the number of singletons to $1$ is $L$.

  \medskip

  Now, we define the input $I_2'$ as a prefix of $I_2$.
  Recall that OPT serves the first external request $(a_k, b_1)$ by placing the component $\langle a_k, b_1\rangle$ in some cluster $D$, and it evicts some nodes $d^{(1)}$ and $d^{(2)}$ from $D$ to the previous locations of $a_k$ and $b_1$.
  It remains to determine the cluster $D$ and nodes $d^{(1)}$ and $d^{(2)}$,
  and our aim is that OPT would not pay any cost for $I_2'$.
  As nodes $d^{(1)}$ and $d^{(2)}$ are not in their initial cluster, a request between a node from (intitial configuration of) $D$ and either $d^{(1)}$ or $d^{(2)}$ would not be free for OPT.
  
  
  At this point we already know the sequence $\langle x^{(i)}\rangle$ of DET.
  For the requests of $I_2'$ to be free for OPT, it suffices to choose $v^D_0$ that is neither $d^{(1)}$ nor $d^{(2)}$ (which is always possible for $k>3$),
  and to guarantee that $d^{(1)}$, $d^{(2)}$ does not appear in the sequence $\langle x^{(i)}\rangle$, i.e., to finish the input sequence before that.
  Furthermore, we would like to guarantee the length of $\langle x^{(i)}\rangle$ to remain $\Omega(k\cdot \ell)$.
  
  To this end, we inspect the sequences $v^{C}_i$ for each cluster, and select such the cluster $D$ whose node $v^{D}_{k-2}$, i.e., the penultimate node appears last in the sequence $\langle x^{(i)}\rangle$, and we end the sequence $I_2'$ just before its appearance.
  Then, we set $d^{(1)} = v^{D}_{k-2}$ and $d^{(2)} = v^{D}_{k-1}$.
  Such choice guarantees that requests from $I_2'$ are free for OPT.
  Furthermore, the number of requests of $I_2'$ is at least $|I_2| - \ell - 1$: as $D$ is the last cluster to remain with two singleton clusters, all other $\ell-1$ clusters have at most $1$ singleton cluster.

  We conclude that for the input $I_1I_2'$ OPT pays $2$, and DET pays at least $2$ for each of $L - \ell - 1 = \Omega(\ell \cdot k)$ requests of $I_2'$.


  \maciek{ToDo: check all off-by-one errors}
  \maciek{ToDo: check if the existence of $b_1$ was considered in all calculations}
  
 % During the runtime of DET, by $x(A)$ we denote the node in the configuration of DET that is co-located in one cluster with the component $a_1, \ldots, a_{k-1}$.

  

 % \maciek{Possibly we need a better name than $x(A)$. This should be tied to (1) DET, and (2) the component $a_1, \ldots, a_{k-1}$ (not the server A, from which it might be relocated by DET). Maybe we should number them, i.e., $x_1, x_2,\ldots$}
 % During the runtime of DET, by $x(A)$ we denote the node in the configuration of DET that is co-located in one cluster with the component $a_1, \ldots, a_{k-1}$.
 % Initially, $x(A) = a_k$.
 
 % We consider a sequence of requests between a node $x(A)$ and some other nodes, i.e., each request concerns the node that is currently co-located with the component $a_1,\ldots,a_{k-1}$.
 % As a result
  

  % Fix a node $m^C$ for each cluster $C$.
  
  % Note that nodes of one cluster are indistinguishible for DET.
  
  
 % For the purpose of constructing $I_2$, we order nodes of any cluster $C$: $v^C_1, \ldots, v^C_k$.
 % For any node $v^C_i\neq v^C_1$ we define $c(v^C_i) = v^C_1$, and $c(v^C_1) = v^C_2$.

  
  
  %Now, for a node $p$ we define $c(p)$ in the following way: for each node outside of $a_i$ and $b$, we define $c(p)$ to be any node that in the initial configuration was placed at the same cluster as $p$, and if $p$ belonged to $D$, then $c(p)$ is any such node that is different from $d_1$ and $d_2$.
  %\maciek{we have a cyclic reference at this point. D is not yet defined (depends on c), and c depends on d1, d2...}
  %Then, the adversary continues to issue requests $(x, c(x))$.
\end{proof}



\section{An $O(kl)$-competitive Algorithm For the Scenario With the Perfect Partition}

\subsection{On a Cost of a Single Repartition}


\subsubsection{A Costly Repartition Example}

A repartition cost is bounded by $O(kl)$ (cite marcin-chen).
Now we show that the rebalance may be expensive (cubic) in terms of $k$, even for $l$ linear in $k$.

This justifies taking a different approach to repartitioning (see Section PPLA).

\maciek{Possibly simplify $\Omega(k^3)$? Or just present $\Omega(k^2)$?}

\subsubsection{For Constant k, Repartition Cost is Constant?}

\maciek{Interesting research problem: is a single repartition cost bounded by $O(poly(k))$? This would improve currently best-known $O(k^2\ell^2)$ to $O(k\cdot poly(k) \cdot \ell)$, i.e., would tighten the lower bound in terms of $\ell$.}
  \maciek{We have such examples for $k=3,4,5$. It gives us an $O(\ell)$-competitive algorithm for $k=3$}

\begin{theorem}
  The rebalancing cost for the greedy algorithm for $k=3$ is $O(1)$.
  \maciek{Greedy = find the cheapest (closest to the current) feasible partition}
  \label{rebalancing-cost}
\end{theorem}

\begin{proof} 
 \maciek{Note: we consider a cost of cheapest feasible rebalancing after insertion of one edge. This in contrast to bounding the distance between any two reconfigurations (this theorem does not bound the latter).}
  
  We distinguish among three configuration types of algorithm's clusters: $C_1, C_2, C_3$. In $C_1$ we have $3$ singleton components (this is also the initial configuration of any cluster). In $C_2$ we have one component of size $2$ and one component of size $1$. Finally, in $C_3$ we have one component of size $3$.

  Consider a request $(u, v)$ and let $U, V$ be their clusters.
  If $U=V$, then the request does not require a reconfiguration.
  Now, we consider cases upon the type of clusters $U$ and $V$.
  Note that this is impossible that either $U$ or $V$ is $C_3$ as otherwise the resulting component would have the size $4$, and this contradicts the existence of the perfect partition.
  If either $U$ or $V$ is $C_1$, then either cluster can fit the merged component, and the rebalance is local within $U$ and $V$, for the cost of at most $3$ migrations.

  Now, we focus on the case where both $U$ and $V$ are $C_2$. Note that $(u,v)$ cannot both belong to a component of size $2$, as the merged component would have size $4$.
  If either $u$ or $v$ belongs to a component of size $2$ then it suffices to exchange components of size $1$ between $U$ and $V$.
  Finally, if $u$ and $v$ belong to components of size $1$, then we must place them in a cluster different from $U$ and $V$.
  Note that in such case, a $C_1$-type cluster exists, as otherwise the input graph would not have a perfect partition. The cost of rebalancing is then at most $12$ \maciek{can show smaller constant than 12, but it does not matter that much}.
\end{proof}

\begin{theorem}
  There exists a $O(\ell)$-competitive algorithm for $k=3$.
\end{theorem}

\begin{proof}
  The adversary issues at most $3\ell$ requests, as otherwise the perfect partition would not exist. Each request issues at most one rebalancingfor the greedy algorithm, and by Theorem~\ref{rebalancing-cost}, the total cost of any algorithm is $O(k\cdot \ell)$.
\end{proof}



\subsection{Perfect Partition Learner Algorithm}


\maciek{$. \rightarrow \cdot$}
\maciek{Font for OPT, ALG, DET, dist etc}

Let $P_I = I_1, \dots, I_{\ell}$ denote the initial partition with which both ON and OPT begin and
$P_F = F_1, \dots, F_{\ell}$ the final partition.
\begin{definition}	\label{def:dist}
	The \emph{distance} of a partition $P = C_1, \dots, C_{\ell}$ is the number of nodes in $P$ that do not reside in their initial cluster.
	That is,
	$dist(P, P_I) = \sum_{j=1}^{\ell} | C_j \setminus I_j |$. 
\end{definition}

In other words,
at least $dist(P, P_I)/2$ node swaps are required in order to reach the partition $P$ from $P_I$.
Therefore,
$OPT \geq \Delta:= dist(P_F, P_I) $.
We devise an online algorithm that mimics OPT by minimizing the distance to the initial partition $P_I$ with each re-partitioning.
As a result,
ON never ends up in a partition that is further away from $P_I$ than $\Delta$.
This invariant ensures that ON does not pay too much while reaching $P_F$.

\begin{algorithm}
	\renewcommand{\algorithmicrequire}{\textbf{Input:}}
	\renewcommand{\algorithmicensure}{\textbf{Output:}}
	\begin{algorithmic}[1]
		\Require 
		$k, \ell$,
		initial partition $P_I$,
		sequence of  edges $\sigma_1, \dots, \sigma_N$ covering all ground truth components
		\Ensure Final partition $P_F$ 
		\State for each node $v$ create a singleton component $C_v$ and add it to $\mathcal{C}$ \label{line:initcomponents}
		\State on communication request $\sigma_t=\{u,v\}$:
		\State Let $C_1 \ni u$ and $C_2 \ni v$ be the container components
		\If{$C_1 \neq C_2$}
		\State unite the two components into a single component $C'$ and
		$\mathcal{C} = (\mathcal{C}\setminus\set{C_1, C_2}) \cup ~\set{C'}$ \label{line:mergecomponents}
		\If{cluster$(C_1) \neq$ cluster$(C_2)$}		
		\State \textit{re-partition($k, \ell, P_I, \mathcal{C}$)} \label{line:rebalance} 
		\EndIf
		\EndIf
	\end{algorithmic}
	\caption{Perfect Partition Learner}
	\label{alg:ppl}
      \end{algorithm}

      \maciek{``re-partition'' procedure name should indicate the fact that this is a specific repartition that is close to initial configuration}

\begin{property} \label{prop:dist<OPT}
	Let $P$ be any partition chosen by Algorithm \ref{alg:ppl} at Line $\ref{line:rebalance}$.
	Then, $dist(P,P_I) \leq \Delta$.
\end{property}

\begin{lemma}	\label{lemma:rebalancecost}
	The cost of re-partitioning at Line \ref{line:rebalance} is at most $2.OPT$.
\end{lemma}
\begin{IEEEproof}
	Consider the re-partitioning that transforms $P_{t-1}$ to $P_t$ upon the request $\sigma_t$.
	Let $M \subset V$ denote the set of nodes that are migrated during this process.
	We have $M = M^+ \cup M^- \cup M^\circ$,
	where $M^-$ ($M^+$) denotes the subset of nodes that
	enter (leave) their original cluster during the re-partitioning,
	and $M^\circ$ denotes the set of remaining nodes in $M$.
	Since $|M^- \cup M^\circ|$ nodes are not in their original cluster before the re-partitioning,
	by Definition \ref{def:dist},
	the distance before the re-partitioning is $dist(P_{t-1}) \geq | M^- \cup M^\circ |$.
	Analogously,
	 the distance afterwards is $dist(P_{t}) \geq | M^+ \cup M^\circ |$.
	Thus,
	$|M| \leq dist(P_{t-1}) + dist(P_{t})$.
	By Property \ref{prop:dist<OPT},
	$dist(P_{t-1}) , dist(P_{t}) \leq \Delta \leq OPT$
	and thereby we have	
	$|M| \leq 2.OPT$.
\end{IEEEproof}

\begin{theorem}	\label{thm:upperbound}
	Algorithm \ref{alg:ppl} reaches the final partition while being $(2.k.\ell)$-competitive.
\end{theorem}
\begin{IEEEproof}
	The algorithm eventually reaches the final partition since it
	 evaluates all possible $\ell$-way partitioning of components on each external request,
	including the request that completes revealing of all ground truth components.
	There are at most $(k-1).\ell < k.\ell $ calls to \emph{re-partition()} and by Lemma \ref{lemma:rebalancecost} each costs at most $2.OPT$.
	The total cost is therefore at most $2.OPT.k.\ell$ which implies the competitive ratio.
\end{IEEEproof}

\appendix

\input{appendix}
	
\end{document}