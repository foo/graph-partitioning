\section{Lower bound $\Omega(\ell)$}

Model assumes the existence of a perfect partition of the input graph.
We present a lower bound for the competitive ratio of $\Omega(\ell)$ for any deterministic algorithm.
To this end, we use $k = 3$, i.e., the capacity of each server is $3$.


\begin{theorem}
  The competitive ratio of any deterministic algorithm for Online Balanced Partition (with existance of a perfect partition of the input graph) is $\Omega(\ell)$.
\end{theorem}
%Note that in the scenario with a perfect partition, both OPT and ALG always pays at most $O(\ell)$.

\begin{proof}
We say that nodes that belong to the same cluster in the initial configuration \emph{have the same color}.

Consider clusters $A$ and $B$ that contain nodes $a_1, a_2, a_3$ and $b_1, b_2, b_3$, respectively.
The adversary begins with issuing three requests: $(a_1, a_2)$, $(b_1, b_2)$, and $(a_3, b_3)$.
Note that these require moving one of these pairs to a third cluster.
Our idea is that the OPT pays for the request $(a_3, b_3)$ only (i.e., it moves this pair to some cluster $C$), and all other requests are free for the OPT.

The algorithm must place these three components in different clusters. Let's call the cluster of $(a_1, a_2)$ the cluster $A_{ALG}$, the cluster of $(b_1, b_2)$ the cluster $B_{ALG}$ and the cluster of $(a_3, b_3)$ the cluster $C$. The configuration of ALG's clusters is then: $A$ contains $a_1, a_2,x$, $B$ contains $b_1, b_2,y$ and $C$ contains $a_3, b_3, z$, for some $x, y, z$ (of choice of the algorithm).
For this, ALG pays at least $2$.

The OPT places these three components in clusters $A, B and D$, paying $2$ for migrating $a_3$ and $b_3$. The OPT has the freedom of choice of the cluster $D$ depending on the choices of the algorithm. We choose $D$ in such way that it does not contain neither $x$ nor $y$, and furthermore that the algorithm places any of nodes from $D$ at $x$ or $y$'s place as the last possible move.
The configuration of OPT's clusters is then: $A$ contains $a_1, a_2,d_1$, $B$ contains $b_1, b_2,d_2$ and $D$ contains $a_3, b_3, d_3$.

From this point, OPT does not change its configuration, and the adversary continues to issue a request between $x$ and any other node with the same color as $x$.
Upon receiving this request, ALG pays for it the cost $1$ and it increases its number of clusters that contain non-singleton components.
This procedure ends when ALG has no more clusters with singleton-only components left or ALG manages to put $d_1, d_2$ or $d_3$ in place of $x$ or $y$.

At the begining, ALG has $\ell$ singleton-only components and each request increases their number by $1$.
Furthermore, before reaching either $d_1, d_2$ or $d_3$ in place of $x$ or $y$, the algorithm must cycle through $\Omega(\ell)$ nodes from other clusters. This is guaranteed as the choice of $D$ can depend on the actions of the deterministic algorithm.
In total, OPT pays $2$, and ALG pays $\Omega(\ell)$.


%The next action of the adversary depends on the choice of $x, y, z$.
%If $x$ and $y$ have different color (i.e., initially they belong to different clusters), then we issue a request between $x$ and any other node with the same color as $x$. As $x$ do not intersect with $D$, the OPT has them co-located in one server, and it pays no cost for such requests. We issue such requests until $x$ and $y$ have the same color.

%Note that the adversary can foresee the actions of the algorithm, and may choose $D$ in such way, that $x$ cycles through all other clusters before reaching $D$, paying the cost of $\Omega(k)$, which would already suffice to show the lower bound.

%Then, $x$ and $y$ have the same color, and we issue a request $(x, y)$. As these have the same color (and there are different than $d_1, d_2$ and $d_3$), OPT pays no cost for this request, and ALG pays $2$ for migrating $x$ and $y$ to other cluster.
%We repeat the entire process, i.e., we force ALG to choose $x$ and $y$ of the same color, and issue such request that is free for OPT. This process can continue until ALG has no more clusters with singleton components, or it pays the cost $\Omega(k)$ already.

\end{proof}

\section{An $O(\ell)$-competitive algorithm for $k=3$}

\begin{theorem}
  The cost of rebalancing cost for algorithm is $O(k)$.
  \label{rebalancing-cost}
\end{theorem}

\begin{proof}
  \maciek{Note: we consider a cost of cheapest feasible rebalancing after insertion of one edge. This in contrast to bounding the distance between any two reconfigurations (we do not do that).}
  
  We distinguish among three configuration types of algorithm's clusters: $C_1, C_2, C_3$. In $C_1$ we have $3$ singleton components (this is also the initial configuration of any cluster). In $C_2$ we have one component of size $2$ and one component of size $1$. Finally, in $C_3$ we have one component of size $3$.

  Consider a request $(u, v)$ and let $U, V$ be their clusters.
  If $U=V$, then the request does not require a reconfiguration.
  Now, we consider cases upon the type of clusters $U$ and $V$.
  Note that this is impossible that either $U$ or $V$ is $C_3$ as otherwise the resulting component would have the size $4$, and this contradicts the existence of the perfect partition.
  If either $U$ or $V$ is $C_1$, then either cluster can fit the merged component, and the rebalance is local within $U$ and $V$, for the cost of at most $3$ migrations.

  Now, we focus on the case where both $U$ and $V$ are $C_2$. Note that $(u,v)$ cannot both belong to a component of size $2$, as the merged component would have size $4$.
  If either $u$ or $v$ belongs to a component of size $2$ then it suffices to exchange components of size $1$ between $U$ and $V$.
  Finally, if $u$ and $v$ belong to components of size $1$, then we must place them in a cluster different from $U$ and $V$.
  Note that in such case, a $C_1$-type cluster exists, as otherwise the input graph would not have a perfect partition. The cost of rebalancing is then at most $12$ \maciek{can show smaller constant than 12, but it does not matter that much}.
\end{proof}

\begin{theorem}
  There exists a $O(\ell)$-competitive algorithm for $k=3$.
\end{theorem}

\begin{proof}
  The adversary issues at most $3\ell$ requests, as otherwise the perfect partition would not exist. Each request issues at most one rebalancing, and by Theorem~\ref{rebalancing-cost}, the total cost of any algorithm is $O(k\cdot \ell)$.
\end{proof}