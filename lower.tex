\section{Warm-up: lower-bound $\Omega(\ell)$ for $k=3$}

\maciek{The intention of this section is to present a simpler version of a lower bound (that should be skipped if we are short on space) AND to present a lower bound for the case with augmentation and discuss the gap}

\begin{theorem}
  The competitive ratio of any deterministic algorithm for Online Balanced Partition with $k=3$ is $\Omega(\ell)$.
  \label{th:lb_omega_l}
\end{theorem}

\begin{proof}
  \maciek{ToDo}
\end{proof}

\begin{observation}
  The lower bound of $\Omega(\ell)$ can be generalized for arbitrary $k = c\cdot 3$.
  The adversary starts with a serie of ``glueing requests'', so that each cluster contains 3 components of size $k/3$.
  We repeat the argument of Theorem~\ref{th:lb_omega_l} using these glued components instead of singletons.
  OPT pays $O(k)$ for migration of one such component, and ALG pays $\Omega(k \cdot \ell)$.
  \label{obs:lb_general_k}
\end{observation}

\begin{observation}
  The lower bound of $\Omega(\ell)$ holds even with $1/3-1/k$-augmentation, i.e., in scenarios where the capacity of each cluster of the algorithm is $(1+1/3)\cdot k - 1$, while the capacity of OPT's clusters is $k$.
  To this end we use the construction from Observation~\ref{obs:lb_general_k}.
  We observe that each cluster of the algorithm can fit exactly $3$ components of size $k/3$.
\end{observation}


  \maciek{Note that this does not contradict the existence of $O((1+1/\epsilon)\cdot k \cdot \log(k))$ algorithm of Avin, Bienkowski et al, as the algorithm requires $2+\epsilon$-augmentation.}

  \maciek{Note that we have a gap here. What is the maximum augmentation (between 1+1/3 and 2) that still has the $\Omega(\ell)$ lower bound?}

\section{Lower bound $\Omega(k\cdot \ell)$}

\maciek{I assume $\alpha = 1$ to simplify calculations, but everything should hold for general $\alpha$}

In this section we consider the Online Balanced Partition problem.
We show a lower bound of $\Omega(k \cdot \ell)$ that uses an input that has a perfect partition.
Therefore, this result is a lower bound for the general Online Balanced Partition problem, as well as a lower bound for Online Balanced Partition with perfect partition.
The latter complements an upper bound $O(k \cdot \ell)$ for Online Balanced Partition with perfect partition that we present in Section \ref{sec:upperbound}.

\begin{theorem}
  The competitive ratio of any deterministic algorithm for Online Balanced Partition is $\Omega(k\cdot \ell)$.
\end{theorem}

\begin{proof}
A simpler proof: \\
We construct an instance of the problem with $\ell$ clusters 
$\set{ S_1, S_2,\dots , S_{\ell}}, |S_i|  = k$.
Let $I(C)$ denote the cluster where nodes in a component $C$ are located initially.
Consider any online algorithm ALG.
We construct the input sequence of requests for ALG as follows.
First,
we issue $k-2$ (internal) requests so that ALG form a component of $k-1$
nodes on clusters $S_1$.
Note that this does not cost ALG.
Let $x_0$  be the only single nodes left on $S_1$ and  $y_0 \in S_2$ be any single node on $S_2$.
Next,
we issue the request between $x_0$ and $y_0$.
Since this is an external request,
ALG joins the into one component and collocates them in some cluster other than $S_1$ and $S_2$.
For this,
ALG moves to a new configuration that replaces $x_0$ and $y_0$ with two other single nodes $x_1$ and $y_1$ respectively.
Next,
we issue a request between $x_1$ and the largest component $C$ s.t.~$I(C) = I(v)$.
ALG must collocate $x_1$ and $C$ on some cluster other than $S_1$ and
consequently replaces $x_1$.
We repeat issuing requests between the single node $x_i$ on $S_1$ and the largest component $C'$ s.t.~$I(C')=I(x_i)$,
 until there are only two single nodes left that  originate from the same cluster.
I.e.,
two single nodes $x^*, y^*,I(x^*) = I(y^*)$
that constitute the last pair of such nodes.
At this point there are at most $\ell+1$ single nodes left
since there must be at least two nodes with the same initial cluster in any subset of $\ell+1$
nodes.
Given this sequence of requests,
the optimal strategy is to migrate $\set{x_0,y_0}$ to the cluster $I(x^*)$ by
 swapping $x_0$ and $y_0$ with $x^*$ and $y^*$ respectively.
Hence,
OPT pays $4$ for node migrations and
ALG pays at least one for each node in the sequence $X := x_0, x_1,\dots$.
We exclude at most $(k-1) + ( \ell+1)$ nodes out of $k.\ell$ nodes,
therefore $|X| \geq k.\ell - k - \ell \in \Omega(k.\ell)$.
\end{proof}

\begin{proof}

  \maciek{Warning! This proof requires still major rewrite before submission!}
  \maciek{The argument of connecting the length of sequence and the number of merged singletons should be stated as an auxiliary lemma.}
  \maciek{Maybe we do not need to construct $I_2$ and then $I_2'$: we can construct $I_2'$ on-the-fly and bound its length. However, the subtracting argument (as it is written now) is quite simple.}
  \maciek{ToDo: (one) figure containing: input $I_1$, cluster $D$, growing components $v^C_i$, and the ``searched area (by ALG)'' (in searching for $d_1, d_2$)}

  \maciek{ToDo: check all off-by-one errors, especially this one: check if the existence of $b_1$ in one of the clusters was considered in all calculations}
 
 
  Fix a component-respecting deterministic algorithm DET.
  Fix a cluster $A$ containing nodes $a_1, \ldots, a_k$ and a cluster $B$ containing (among others) a node $b_1$.
  The adversary begins with issuing $k-2$ internal requests $(a_i, a_{i+1})$, so that nodes $a_1, \ldots a_{k-1}$ form a single component.
  These internal requests are free for OPT, and it does not perform additional actions at this point.
  The next request issued by the adversary is an external request $(a_k, b_1)$. %, which at this point is the request $(a_k, b)$.
  Note that DET must relocate the component $\langle a_k, b_1\rangle$ to some other cluster, as otherwise it is not component-respecting.
  OPT also relocates the component $\{ a_k, b_1\}$ to some carefuly chosen cluster $D$, and it evicts nodes $d^{(1)}$ and $d^{(2)}$ (to be fixed later) from $D$ to the previous locations of $a_k$ and $b_1$ (i.e., to the clusters $A$ and $B$).
  For these actions, OPT pays $2$, and DET pays at least $2$.
  We call the requests issued so far the requests $I_1$.


  Now, we construct a sequence of requests $I_2$, but we do not feed them to DET yet.
  Instead, we simulate DET's actions on $I_1I_2$ to decide at which point to terminate the input sequence earlier, i.e., we choose a prefix $I_2'$ of $I_2$ and we feed DET the input $I_1I_2'$.
  %\maciek{We can produce $I_2'$ on-the-fly, but this was convinient to write it down this way}


  The input $I_2$ consists of requests $(x^{(1)}, c(x^{(1)})), (x^{(2)}, c(x^{(2)})),\ldots$.
  The node $x^{(i)}$ is a node in the current configuration of DET that is co-located in one cluster with the component $a_1, \ldots, a_{k-1}$.
  After a request $(x^{(i)}, c(x^{(i)}))$, the node $x^{(i)}$ is a part of a non-singleton component and must be relocated to some other cluster (as its current cluster hosts a component $a_1, \ldots, a_{k-1}$).
  Then, DET places the node $x^{(i+1)}$ in place of $x^{(i)}$, and the next request concerns that node.
  % Initially, $x^{(1)} = a_k$, and the next node $x^{(2)}$ is the that DET locates in place of the
  The sequence continues as long as after serving the request, a perfect partition of the input graph exists.

  For each $x^{(i)}$, we set $c(x^{(i)})$ to some (to-be-determined) node that in the initial configuration was co-located with $x^{(i)}$.
  While producing an input sequence, we would like to additionally order the initial nodes $V^C$ of each cluster (i.e., order the nodes that were present in each cluster in its initial configuration).
  The ordering needs to be produced on-the-fly, as this ordering is used to produce subsequent requests.
  
  %Note that in the initial configuration, nodes of one cluster are indistinguishible for DET \maciek{and hence we can order them on-the-fly}.
  
  Now, we formally define $c(x^{(i)})$, and the ordering of nodes in the initial configuration.
  Consider a set of nodes from an initial configuration of a cluster $C$ (where $C \neq A$).
  By $v^C_1$ we label the first node of a cluster $C$ that appears in the sequence $\langle x^{(i)}\rangle$.
  At this point, we fix an arbitrary other node of $C$ and we label it $v^C_0$, and we set $c(v^C_i) = v^C_0$ for all $i$.
  For the following $k-2$ nodes of the cluster $C$, we label them $v^C_2, v^C_3, \ldots$ in the order of appearance in the sequence $\langle x^{(i)}\rangle$.
  % For a node $v^C_p = x^{(j)}$, we set $y^{(j)} = v^C_0$.
  
  If some nodes of $C$ never appear in the sequence $\langle x^{(i)}\rangle$ (e.g. DET might terminate prematurely or might reside in a configuration with components split)
  , we number them arbitrarily, with the restriction that the nodes that do appear in the sequence have smaller number than the ones that do not \maciek{This should not happen at all for component-respecting DET, though. Would be better to provide an additional lemma that shows that every node of each cluster appears in the sequence (outside of $v_0$'s, $a_1,\ldots,a_{k-1}, a_k, b_1$)}
  
  Now, we claim that given the input $I_1I_2$, every component-respecting DET 
  % (that does not terminate prematurely and never resides in a configuration with components split)
  must produce a sequence $\langle x^{(i)}\rangle$ of length $L = k\cdot \ell - (k-1) - 2 - (\ell-1) - 1$.
  \maciek{ToDo: a sentence that explains each summand of $L$}
  \maciek{Name of $L$ should indicate a connection to $I_2$, e.g., maybe just $|I_2|$, and then $L-\ell-1$ from last paragraph should be related to $I_2'$}
  Consider a request $(x^{(i)}, c(x^{(i)}))$.
  If in this configuration at least one singleton component is present (other than the components of $(x^{(i)}, c(x^{(i)}))$), then serving this request is possible (i.e., the perfect partition exists), and hence a singleton node appears as $x^{(i+1)}$ and the sequence continues.
  Otherwise, no perfect partition exists, as we have a component $\{a_1, \ldots, a_{k-1}\}$ of size $k-1$ and it can only be complemented by a singleton component.

  \maciek{A candidate for auxiliary lemma: intuitions before the claim, then the formal formulation, and a rewrite of the argument from this paragraph}
  Hence, the length of a sequence $\langle x^{(i)}\rangle$ is closely related to the number of singleton components.
  Before the arrival of the first request from $I_2$, the number of singleton components is $L + (\ell-1) + 1$, and after serving the last request, the number of singleton components is $1$.
  %Every node $x^{(i)}$ must be a singleton component, otherwise DET resides in a configuration with components split (and such DETs either leave such configuration eventually or issue an infinite cost).
  After each request, we join this singleton component with another component.
  This join decreases the number of singleton components by either $1$ or $2$: if we join two singletons $\{v^C_0\}, \{v^C_1\}$ (there are $\ell-1$ such joins: for all clusters but $A$), then the number of singletons decrease by $2$, and it decreases by $1$ otherwise.
  We conclude that the number of requests $\langle x^{(i)}\rangle$ to decrease the number of singletons to $1$ is $L$.

  \medskip

  Now, we define the input $I_2'$ as a prefix of $I_2$.
  Recall that OPT serves the first external request $(a_k, b_1)$ by placing the component $\langle a_k, b_1\rangle$ in some cluster $D$, and it evicts some nodes $d^{(1)}$ and $d^{(2)}$ from $D$ to the previous locations of $a_k$ and $b_1$.
  It remains to determine the cluster $D$ and nodes $d^{(1)}$ and $d^{(2)}$,
  and our aim is that OPT would not pay any cost for $I_2'$.
  As nodes $d^{(1)}$ and $d^{(2)}$ are not in their initial cluster, a request between a node from (intitial configuration of) $D$ and either $d^{(1)}$ or $d^{(2)}$ would not be free for OPT.
  
  
  At this point we already know the sequence $\langle x^{(i)}\rangle$ of DET.
  For the requests of $I_2'$ to be free for OPT, it suffices to choose $v^D_0$ that is neither $d^{(1)}$ nor $d^{(2)}$ (which is always possible for $k>3$),
  and to guarantee that $d^{(1)}$, $d^{(2)}$ does not appear in the sequence $\langle x^{(i)}\rangle$, i.e., to finish the input sequence before that.
  Furthermore, we would like to guarantee the length of $\langle x^{(i)}\rangle$ to remain $\Omega(k\cdot \ell)$.
  
  To this end, we inspect the sequences $v^{C}_i$ for each cluster, and select such the cluster $D$ whose node $v^{D}_{k-2}$, i.e., the penultimate node appears last in the sequence $\langle x^{(i)}\rangle$, and we end the sequence $I_2'$ just before its appearance.
  Then, we set $d^{(1)} = v^{D}_{k-2}$ and $d^{(2)} = v^{D}_{k-1}$.
  Such choice guarantees that requests from $I_2'$ are free for OPT.
  Furthermore, the number of requests of $I_2'$ is at least $|I_2| - \ell - 1$: as $D$ is the last cluster to remain with two singleton clusters, all other $\ell-1$ clusters have at most $1$ singleton cluster.

  We conclude that for the input $I_1I_2'$ OPT pays $2$, and DET pays at least $2$ for each of $L - \ell - 1 = \Omega(\ell \cdot k)$ requests of $I_2'$.

 
 % During the runtime of DET, by $x(A)$ we denote the node in the configuration of DET that is co-located in one cluster with the component $a_1, \ldots, a_{k-1}$.

  

 % \maciek{Possibly we need a better name than $x(A)$. This should be tied to (1) DET, and (2) the component $a_1, \ldots, a_{k-1}$ (not the server A, from which it might be relocated by DET). Maybe we should number them, i.e., $x_1, x_2,\ldots$}
 % During the runtime of DET, by $x(A)$ we denote the node in the configuration of DET that is co-located in one cluster with the component $a_1, \ldots, a_{k-1}$.
 % Initially, $x(A) = a_k$.
 
 % We consider a sequence of requests between a node $x(A)$ and some other nodes, i.e., each request concerns the node that is currently co-located with the component $a_1,\ldots,a_{k-1}$.
 % As a result
  

  % Fix a node $m^C$ for each cluster $C$.
  
  % Note that nodes of one cluster are indistinguishible for DET.
  
  
 % For the purpose of constructing $I_2$, we order nodes of any cluster $C$: $v^C_1, \ldots, v^C_k$.
 % For any node $v^C_i\neq v^C_1$ we define $c(v^C_i) = v^C_1$, and $c(v^C_1) = v^C_2$.

  
  
  %Now, for a node $p$ we define $c(p)$ in the following way: for each node outside of $a_i$ and $b$, we define $c(p)$ to be any node that in the initial configuration was placed at the same cluster as $p$, and if $p$ belonged to $D$, then $c(p)$ is any such node that is different from $d_1$ and $d_2$.
  %\maciek{we have a cyclic reference at this point. D is not yet defined (depends on c), and c depends on d1, d2...}
  %Then, the adversary continues to issue requests $(x, c(x))$.
\end{proof}

