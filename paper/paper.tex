 
\documentclass[manuscript,screen=true]{acmart}

\usepackage[utf8]{inputenc}
\usepackage{xspace}
\usepackage{balance}
\usepackage{amsmath,amsfonts,mathtools,amsthm}
\usepackage{algorithmic}
\usepackage{algorithm}

\mathchardef\mhyphen="2D

\title{Brief Announcement: An Improved Lower Bound for Dynamic Balanced Graph Partitioning}


\author{Maciej Pacut}
\email{maciek.pacut@gmail.com}
\orcid{0000-0002-6379-1490}
\affiliation{%
  \institution{Faculty of Computer Science, University of Vienna}
  \country{Austria}
}


\author{Mahmoud Parham} 
\email{mahmoud.parham@univie.ac.at}
\orcid{0000-0002-6211-077X}
\affiliation{%
  \institution{Faculty of Computer Science, University of Vienna}
  \country{Austria}
}

\author{Stefan Schmid} 
\email{stefan_schmid@univie.ac.at}
\affiliation{%
  \institution{Faculty of Computer Science, University of Vienna}
  \country{Austria}
}

\copyrightyear{2020} 
\acmYear{2020} 
\setcopyright{acmlicensed}
\acmConference{PODC '20}{July 29-August 2, 2020}{Toronto, Canada}

\keywords{online algorithms, competitive analysis, graph partitioning, clustering}


%%%%%%%%%%%%%%%%%%%%%%%%%%%%%%%%%%%%%%%%%%%%%%%%&&
%%%%%%%%%%%%%%%%%%%%%%%%%%%%%%%%%%%%%%%%%%%%%%%%&&
%  our macros start
%%%%%%%%%%%%%%%%%%%%%%%%%%%%%%%%%%%%%%%%%%%%%%%%&&
%%%%%%%%%%%%%%%%%%%%%%%%%%%%%%%%%%%%%%%%%%%%%%%%&&

\newcommand{\OPT}{\textsf{OPT}\xspace}
\newcommand{\ALG}{\textsf{ALG}\xspace}
\newcommand{\PPL}{\textsf{PPL}\xspace}
\newcommand{\OBRP}{BRP}
\newcommand{\PPOBRP}{PP-BRP}
\newcommand{\dist}{\textsf{dist}}
\newcommand{\TAlg}{{\ensuremath{\textsf{ALG}_{3}}}\xspace} % we should change this name




\newtheorem{claim}{Claim}
\newtheorem{fact}{Fact}
\newtheorem{rem}{Remark}
\newtheorem{observation}{Observation}
\newtheorem{property}{Property}


\DeclarePairedDelimiter\pair{(}{)}
\DeclarePairedDelimiter\set{\{}{\}}

\DeclarePairedDelimiter{\ceil}{\lceil}{\rceil}
\DeclarePairedDelimiter{\floor}{\lfloor}{\rfloor}

\newcommand\mahmoud[1]{\color{green}\textbf{\\ Mahmoud: #1}\\\color{black}}
\newcommand\stefan[1]{\color{blue}\textbf{\\ Stefan: #1}\color{black}}
\newcommand\maciek[1]{\color{brown}\textbf{\\ Maciek: #1}\color{black}}


\newcommand{\todo}[1]{\noindent\color{brown}{todo: #1}\color{black}}


\begin{document}


\begin{abstract}
  Distributed   applications,  including  batch  processing, streaming, and scale-out databases generate a significant amount of network traffic and a~considerable fraction of their runtime is due to network activity. In this  paper, we study  algorithms  for collocating frequently communicating nodes in a distributed networked systems in an online fashion.

  In the online graph re-partitioning problem, we are given a sequence of pairwise communication requests between n nodes, revealed online.
  The objective is to service these requests by partitioning the nodes into $\ell$ clusters, each of size $k$, such that frequently communicating nodes are located in the same cluster.
  The partitioning can be updated by migrating nodes between clusters for a fixed cost.
  The goal is to jointly minimize the amount of inter-cluster communication and migration cost.


  So far, the best known lower bound for the competitive ratio of
	deterministic algorithms was $\Omega(k)$,
 by reduction from online paging.
  We improve this result significantly by providing the lower bound of $\Omega(k\cdot \ell)$.
  We follow up with an optimal algorithm for $k=3$.
  Finally,
  we provide a $O(k\cdot \ell)$-competitive algorithm for the variant
  where the optimal configuration is always a fixed partitioning (a.k.a~a \emph{configuration})
  that incurs zero communication cost through the whole request sequence.
\end{abstract}
    
\maketitle
    
\renewcommand{\shortauthors}{M.~Pacut, M.~Parham, S.~Schmid}

\section{Introduction}


Main technical contribution of this paper is the lower bound $\Omega(k\cdot\ell)$ for the competitive ratio of any deterministic algorithm for online balanced re-partition problem (see Section~\ref{sec:lowerbound}).
The best known lower bound so far was $\Omega(k)$ \cite{repartition-disc}
which involves only two clusters.


So far, the competitive ratio of the best known algorithm was the component-based algorithm with the ratio $O(k^2\cdot \ell^2)$~\cite{repartition-disc}.
By applying classic rent-or-buy technique, we obtained an optimal algorithm for $k=3$, that we present in Section~\ref{sec:k3}.

In Section~\ref{sec:ppl}, we study a restricted model of balanced re-partition, called perfect-partition.
For that, we present a $O(k\cdot \ell)$-competitive algorithm, which is optimal.
We believe that this result is an important building block on the way to an optimal algorithm for the general model.

\section{Related Work}

The problem was studied under resource augmentation \cite{repartition-disc}, and under stochastic assumptions of input\maciek{cite}.\maciek{cite other papers that cite DISC?}

The model is related to online
caching~\cite{SleTar85,FKLMSY91,McGSle91,AcChNo00}. In our
setting, requests can be served remotely, which is
reminiscent of caching models \emph{with
bypassing}~\cite{EpImLN11,EpImLN15,Irani02}.


The static offline version of~the~problem, called the
\emph{$\ell$-balanced graph partitioning problem} is 
NP-complete, and cannot even be approximated within any finite factor unless P
= NP~\cite{AndRae06}. The static
variant where $\ell = 2$ corresponds to the minimum bisection problem, which
is already NP-hard~\cite{GaJoSt76}, and 
the currently best approximation ratio is $O(\log n)$~\cite{SarVaz95,ArKaKa99,FeKrNi00,FeiKra02,KraFei06,Raec08}.




\section{Preliminaries}
\label{sec:problem-definition}

The online \emph{balanced re-partitioning} problem (\OBRP{}) is defined as
follows. We are given a set of $n$ nodes,
initially arbitrarily partitioned into $\ell$~clusters,
each of size~$k$.
The input consists of a sequence of pairwise communication requests
$\sigma = (u_1,v_1),$ $(u_2,v_2),$ $(u_3,v_3), \ldots$,
where a pair $(u_t,v_t)$ indicates that nodes $u_t$ and $v_t$ exchange a fixed amount of data.
Nodes in $C \subset V$ are \emph{collocated}
if they reside in the same cluster.
An algorithm serves a communication request between two nodes
either \emph{locally} at cost~0
if they are collocated,
or \emph{remotely} at cost~1
if they are located in different clusters.
We refer to the two types of requests as \emph{internal}
and \emph{external} requests, respectively.
Before serving a request,
an online algorithm may \emph{re-partition},
%(i.e., \emph{reconfigure}).
i.e.,
it may (optionally) move some nodes into clusters different than their current clusters while respecting the capacity of every cluster. 
Only after a re-partitioning is completed,
the algorithm serves the  request.
The cost of migrating a node from one cluster to another
is~$\alpha \in \mathbb{Z}^+$.
For any algorithm $ALG$,
its cost,
denoted by $ALG(\sigma)$,
is the total cost of communications and
the cost of migrations performed by $ALG$ while serving the sequence $\sigma$.


%\maciek{I just copied useful sentences. This section is TODO}
%An algorithm that  keeps all components collocated is \emph{component-respecting}.
%
A \emph{component} is a subset of frequently communicating nodes.
Initially,
every node is trivially a \emph{singleton} component.
We grow components similar to previous approaches~\cite{repartition-disc},
which proved to be an effective technique for mimicking an optimal offline algorithm.
An algorithm is \emph{component-respecting}
if it always keeps nodes that  belong to the same component collocated.
That is,
if the algorithm needs to move a node,
it moves the whole component containing the node.
%We say that the request $(u,v)$ is \emph{external}, if the algorithm is in the configuration, where $u$ and $v$ are located at different clusters.
%This is achieved by opting a sequence of  partitions (a.k.a.~configurations)
%such that frequently communicating nodes (from different clusters)
%are relocated to the same cluster before they inflict too much communication cost.
%A \emph{reconfiguration} (i.e.,\emph{ re-partitioning})
%is performed by migrating nodes between clusters,
%at the total cost of individual node migrations.
The objective is to jointly minimize communication and migration costs while serving the sequence.


\section{Lower bound} %$\Omega(k\cdot \ell)$ for Competitive Ratio of Any Deterministic Algorithm}
\label{sec:lowerbound}


\maciek{Here, we would like to introduce insights about the proof, emphasize the conditions for the bound to work (such as $k\geq 3$), and maybe compare to previously known lower bounds. Also, introduce the notation and definitions that are needed.}
\maciek{We should describe what we mean by  components (that we force alg to keep on the same cluster). We use components differently in $k=3$, but we denote them by components, too. This is understandable, but must be emphasized.}

\begin{theorem}
	The competitive ratio of any deterministic algorithm for \OBRP{} is in $\Omega(k\cdot \ell)$ for any $k\geq 3$.
\end{theorem}

\begin{proof}
	We construct an instance of the problem with $\ell$ clusters 
	$\set{ S_1, S_2,\dots , S_{\ell}}, |S_i|  = k$.
	Let $I(C)$ denote the cluster where nodes in a component $C$ are located initially.
	%	\maciek{So far components were not introduced; the best place to introduce these would be around the definition of \PPOBRP}.
	Fix any online algorithm \ALG{}.
	We construct the input sequence for \ALG{} as follows.
	First,
	we issue $k-2$ (internal) requests so that \ALG{} can form a component of $k-1$
	nodes on the cluster $S_1$.
	Recall that internal requests does not incur any cost for \ALG{}.
	If \ALG{} at any point splits a component
	(i.e., spreads its nodes over two or more clusters),
	then we continue to issue requests between non-collocated nodes of the component until \ALG{} collocates the nodes of the component.
	Hence, if \ALG{} never collocates them then it is not competitive.
  
  \maciek{Components are not introduced. Thus, the reader does not know what a single node means.}
	Let $x_0$  be the only single node left on $S_1$ and  $y_0 \in S_2$ be any single node on $S_2$.
	Next,
	we issue a request between $x_0$ and $y_0$.
	Since this is an external request,
	\ALG{} joins them into one component and collocates them in some cluster other than $S_1$.
	For this,
	\ALG{} moves to a new configuration
	that replaces $x_0$ and $y_0$ with two other single nodes $x_1$ and $y_1$ respectively.
	
	Next,
	we issue a request between $x_1$ and the largest component $C$ s.t.~$I(C) = I(x_1)$.
	\ALG{} must collocate $x_1$ and $C$ on some cluster other than $S_1$ and
	consequently replaces $x_1$ with some other (single) node $x_2$.
	We repeat issuing requests between the single node $x_i$ on $S_1$ and the largest component $C'$ s.t.~$I(C')=I(x_i)$,
  until there are only two single nodes left that  originate from the same cluster.
	Formally, we denote these two single nodes by $x^*, y^*$, and we have $I(x^*) = I(y^*)$, and~for any other pair of single nodes
	$x'$ and $y'$,
	we have $I(x') \neq I(y')$.
	At this point there are at most $\ell+1$ single nodes left,
	otherwise there would be more pairs of single nodes that were initially in the same cluster.
	
	Given this sequence of requests,
	the optimal strategy is to migrate $\set{x_0,y_0}$ to the cluster $I(x^*)$ by
	swapping $\set{x_0,y_0}$ with $\set{x^*,y^*}$.
	Hence,
	OPT pays for $4$ node migrations and
	\ALG{} incurs at least one migration for each node in the sequence $X := x_0, x_1,\dots$.
	We exclude at most $(k-1) + ( \ell+1)$ nodes out of $k \cdot \ell$ nodes,
	therefore $|X| \geq k \cdot \ell - k - \ell \in \Omega(k\cdot\ell)$.
\end{proof}


\section{An Optimal Algorithm for Clusters of Size 3}
\label{sec:k3}

\maciek{TODO: find better source for rent-or-buy, e.g. ski rental paper}

In this section we show that the classic \emph{rent-or-buy} approach\cite{borodin-book} allows to obtain an optimal algorithm for $k=3$.
Upon receiving $\alpha$ requests between a pair of nodes, we collocate them in one cluster until the end of a phase (that we define precisely later).



%We define the algorithm \TAlg in the following way.
Now we describe the algorithm \TAlg.
It partitions nodes into components, and
initially each node belongs to its own singleton component.
For each pair of nodes, \TAlg maintains a counter, initiated to $0$. 
Upon receiving a request to a pair that is not collocated at one cluster, it increases their counter by $1$.
If the counter for a pair $(u,v)$ reaches $\alpha$, \TAlg merges the components of $u$ and $v$, and moves to the closest component-respecting partitioning.
If no such partitioning exists, \TAlg resets all components to singletons, resets all counters to $0$, and ends the phase.
\maciek{Serving remotely: say explicitly}

%\maciek{Shouldn't we think about whole component cuts instead of single edge? Hmm. NO!}



In our analysis, we distinguish among three types of clusters: $C_1, C_2, C_3$. In a cluster of type $C_i$, the size of the largest component contained in this cluster is $i$.

\begin{lemma}
  \label{lem:1req}
  A single re-partition of \TAlg consists of at most $2$ migrations.
  \maciek{todo:propagate the optimization of cost}
\end{lemma}

\begin{proof}
  Observe that when the re-partition is triggered by \TAlg, the resulting partitioning is component-respecting.
  Otherwise, if it does not exist, \TAlg simply ends the phase and performs no re-partition.

  %We distinguish between three types of clusters: $C_1, C_2, C_3$,
  %which we define as follows.
  %A cluster of type $C_1$ the cluster contains $3$ singleton components (this is also the initial configuration of any cluster).
  %A cluster of type $C_2$  contains one component of size $2$ and one component of size $1$.
  %Finally, a cluster of type $C_3$  contains one component of size $3$.
  
    Consider a request between $u$ and $v$ that triggered the re-partition and let $U$ and $V$ be their respective clusters.
    Note that $U\neq V$,
	 as otherwise the request would not trigger a~re-partitioning.
	 We consider cases upon types of clusters $U$ and $V$.
    Note that this is impossible that either $U$ or $V$ is of type $C_3$, as otherwise we merge components of size $3$, and no component-respecting partition exists, a contradiction.
    If either $U$ or $V$ is of type $C_1$, then this cluster can fit the merged component, and the reconfiguration is local within $U$ and $V$.
    For two clusters, any configuration can be reached within two migrations, due to the fact that clusters are indistiguishable.
  
    Finally, we consider the case where both $U$ and $V$ are of type $C_2$. Note that $(u,v)$ cannot both belong to a component of size $2$, as this would mean that \TAlg has the component of size $4$, a contradiction with the case assumption that the component-respecting partitioning exists. 
    Otherwise, if either $u$ or $v$ belongs to a component of size $2$, then it suffices to exchange components of size $1$ between $U$ and $V$.
    Finally, if $u$ and $v$ belong to components of size $1$, then we must place them in a cluster different from $U$ and $V$.
    Note that in such case, a $C_1$-type cluster $W$ exists, as otherwise no component-respecting partition exists. In this case \TAlg performs one migration, i.e., it exchanges the nodes $u$ and $v$ with any two nodes of $W$.
\end{proof}



\begin{theorem}
  \TAlg is $O(\ell)$-competitive.
\end{theorem}
\begin{proof}
  Fix a completed phase, and consider the state of \TAlg's counters at the end of it.
  We consider the incomplete phase later in this proof.

  As \TAlg is component-respecting, it never increases any counter above $\alpha$.
  We say that the pair $(u, v)$ is \emph{saturated}, if the counter has value $\alpha$, and \emph{unsaturated} otherwise.


  Let $\sigma$ be the input sequence that arrived during the phase.
  Let $\sigma_{cost}$ be the requests that at the moment of arrival were external requests for \TAlg (these are the only requests that incurred a cost for \TAlg).
  In our analysis, we partition $\sigma_{cost}$ into subsequences $\sigma_I$ and $\sigma_E$.
  The sequence $\sigma_I$ (inter-component requests) are the requests from $\sigma_{cost}$ issued to pairs that belong to the same component of \TAlg at the end of the phase.
  The sequence $\sigma_E$ (extra-component requests) denotes the requests from $\sigma_{cost}$ that do not appear in $\sigma_I$.


  %Let $A^I$ be the cost of (extra-cluster) communication incurred in this phase by \TAlg between pairs that belonged to the single component at the end of the phase.
  %Let $A^E$ be the cost of (extra-cluster) communication incurred in this phase between the nodes that belong to different components at the end of this phase.
  Let $\TAlg(M)$ be the cost of migrations performed by \TAlg in this phase.
  \TAlg performs at most $2 \ell$ component merge operations, as
  exceeding this number means that a component of size $4$ exists, and the phase should have ended already.
  Combining this with Lemma~\ref{lem:1req} gives us $\TAlg(M) \leq 6\alpha\cdot\ell$.
  %Together with Lemma~\ref{lem:1req}, this allows us to bound the cost of migrations, $\TAlg(M) \leq 6\cdot\alpha\cdot\ell$.
  
  Now we bound $\TAlg(\sigma_I)$.
  Cluster of type $C_3$ contributes at most $3 \alpha - 1$ to $\TAlg(\sigma_I)$, as $2$ of pairs of nodes from the component are saturated and contribute $\alpha$ each, and the third, unsaturated pair contributes at most $\alpha-1$.
  Other cluster types contribute less: $C_1$ contributes $0$ and $C_2$ contributes $\alpha$.
  Summing this over all $\ell$ clusters gives us $\TAlg(\sigma_I) \leq (3 \alpha-1)\cdot \ell \leq 4\alpha\cdot\ell$.

  %We bound $A^E$ by $k^2 \cdot (\alpha - 1)$, as no more than $k^2$ pairs are unsaturated, and each of them contributes at most $\alpha -1$.
  %\maciek{not needed most likely}

  Moreover, \TAlg paid for all requests from $\sigma_E$, and thus $\TAlg(\sigma_E) = |\sigma_E|$.
  In total, the cost of \TAlg is at most $\TAlg(\sigma_I) + \TAlg(\sigma_E) + \TAlg(M) \leq 10\alpha\cdot \ell + |\sigma_E|$ during this phase.

  \medskip

  Now we lower-bound the cost of $\OPT$.
  By $\OPT(\sigma_I)$ and $\OPT(\sigma_E)$ we denote the cost of $\OPT$ on these input sequences (defined with respect to components of \TAlg in this phase).
  By $\OPT(M)$ we denote the cost of migrations performed by $\OPT$ in this phase.
  
  We split the cost of $\OPT$ into parts coming from serving $\sigma_I$ and $\sigma_E$.
  While serving these requests, $\OPT$ may perform migrations, and we account for them in both parts: we separately bound $\OPT$ by $\OPT(\sigma_I) + \OPT(M)$ and $\OPT(\sigma_E) + \OPT(M)$.
  Combining those bounds gives us $\OPT \geq \max\{\OPT(\sigma_I) + \OPT(M), \OPT(\sigma_E) + \OPT(M)\} \geq (\OPT(\sigma_I) + \OPT(M)) / 2 + (\OPT(\sigma_E) + \OPT(M)) / 2$.

  %Let $O^M$ be the cost of migrations performed by $\OPT$ during the phase.
  %Let $O^I$ be the cost of serving requests between nodes that were put in one component by \TAlg during this phase.
  %Let $O^E$ be the cost serving requests between nodes that \TAlg did not put in the same component during that phase.

  %First, we estimate the cost related to $\sigma_I$.
  We have $\OPT(M) + \OPT(\sigma_I) \geq \alpha$, as the phase ended when the components of \TAlg{} could not be partitioned without splitting them.
  Hence, for every possible configuration of $\OPT$, there exists a non-collocated pair of nodes with at least $\alpha$ requests between them, and
  $\OPT$ either served them remotely or performed a~migration.

  \medskip
  Before we bound the competitive ratio, we relate the costs of $\TAlg$ and $\OPT$ with respect to requests $\sigma_E$.
  In a fixed configuration of $\OPT$, it may mitigate paying for requests between at most $3\ell$ pairs of nodes by collocating them in its clusters.
  Recall that $\sigma_E$ consists of requests to unsaturated pairs, and it accounts only for requests that increased the counter (i.e., external requests), thus $\OPT$ may mitigate at most $3\ell\cdot(\alpha - 1)$ requests from $\sigma_E$.
  Faced with $W := |\sigma_E| - 3\ell\cdot(\alpha-1)$ requests $\sigma_E$ that it could not mitigate, $\OPT$ served some of them remotely and possibly performed some migrations to decrease its cost.


  Now we estimate the cost of $\OPT(\sigma_E)$ by accounting savings from migrations.
  By performing a swap of nodes $(u,v)$, $\OPT$ collocates $u$ with two nodes $u', u''$, and $v$ with two nodes $v'$, $v''$.
  This may allow to serve requests between $(u,u')$, $(u,u'')$, $(v,v')$ and $(v,v'')$ for free afterwards.
  As $\sigma_E$ consists of requests to unsaturated pairs, and it accounts only for external requests, there are at most $\alpha-1$ requests between each of these pairs.
  By performing a single migration that costs $\alpha$, $\OPT$ may avoid paying the remote serving costs for at most $4 (\alpha - 1)$ requests from $\sigma_E$.
  Thus, for serving $\sigma_E$, $\OPT$ pays at least $W \cdot \frac{\alpha}{4 (\alpha-1)}$.
  This gives us
  \begin{equation*}
    \OPT(\sigma_E) + \OPT(M)  \geq (|\sigma_E|-3(\alpha - 1)\cdot \ell) \cdot \frac{\alpha}{4 (\alpha-1)}
    \geq |\sigma_E| / 4 - 4 \alpha \cdot \ell.
  \end{equation*}
  From that, we obtain $|\sigma_E| \geq 4(\OPT(\sigma_E)+\OPT(M)) + 16\alpha \cdot \ell$.
  Let $E := \OPT(\sigma_E) + \OPT(M)$. Finally, we are ready to bound the competitive ratio
  \begin{equation*}
    \frac{\TAlg(\sigma)}{\OPT(\sigma)} \leq \frac{10\alpha \cdot \ell + |\sigma_E|}{(\alpha + \OPT(\sigma_E) + \OPT(M))/2} \leq \frac{52\alpha\cdot\ell + 8\cdot E}{\alpha + E} \leq 52 \ell = O(\ell).
  \end{equation*}

  \medskip

  It remains to consider the last, unfinished phase.
  Consider the case, where the unfinished phase is also the first one.
  Then, we cannot charge $\OPT$ due to unability to partition the components.
  Instead, we use the fact that \TAlg and $\OPT$ started with the same initial configuration.
  We charge $\OPT$ $\alpha$ for the first external $\alpha$ requests or a migration,
  and we follow the analysis regarding the unsaturated requests.
  If the input finished before first $\alpha$ external requests, then \TAlg is optimal.
  Now, consider the case, where there are at least two phases, then we split the cost $\alpha$ charged in the penultimate phase into last two phases, and follow the analysis regarding the unsaturated requests.
  This way, the competitive ratio increases at most twofold.
\end{proof}


\section{An Optimal Algorithm for Perfect-Partition}	\label{sec:ppl}
%\subsection{Perfect Partition Model}\label{sec:perfectPartition}
%\maciek{Here introduce a restricted model of \PPOBRP{}, the restricted variant where the input sequence can be perfectly partitioned}
%\maciek{ every algorithm must colocate every communicating pair.
%	\mahmoud{updated!}}
%\maciek{Note that the algorithm from Section~\ref{sec:k3} is $O(\ell)$-competitive for the Perfect Partition Model, as during the first phase, a perfect partition of input exists.}

%\subsection{An $O(k\cdot \ell)$-competitive Algorithm for Perfect Partition}

A \emph{perfect-partition} is a partitioning of nodes into $\ell$ clusters
such that it incurs no inter-cluster communication throughout the input sequence.
Equivalently, a perfect-partitioning it an $\ell$-way cut of the communication graph of the input sequence.
In this section,
we give an algorithm for a restricted variant of  \OBRP{},
%\maciek{This is not sufficient! We also disallow serving requests remotely.}
where an optimal offline algorithm ($\OPT$) moves to this perfect-partition
at the beginning and stays there perpetually.
%\maciek{Why do we have to assume that? This is just what OPT does, i.e., we can prove that this strategy (a) lower bounds the OPT's cost and (b) is sufficent for our analysis. But now is sounds like we restrict the model even further.}
We refer this variant as \PPOBRP{}.
The task of an online algorithm for \PPOBRP{} is to recover (or learn) the perfect-partition while not paying too much relative to \OPT.
%
%\maciek{This is not a valid description of the model. We need the assumption that collocation is needed in our model. Otherwise OPT would just serve some requests remotely, and our analysis would not hold.}
The first request between any two non-collocated nodes reveals an edge of the communication graph.
Our online algorithm collocates them immediately and never separates them since by assumption \OPT does the same.
Revealed edges, up to any point in the sequence,
induce some connected subgraphs and we refer to them as \emph{components}.
An algorithm that always  keeps nodes belonging to the same component collocated is \emph{component-respecting}.

\subsection{Perfect-Partition Learner (\PPL) Algorithm}
     
We assume \OPT begins with the initial configuration
$P_I = I_1, \dots, I_{\ell}$ and moves to the final partitioning
$P_F = F_1, \dots, F_{\ell}$.
 The \emph{distance} of a configuration $P = C_1, \dots, C_{\ell}$ from the initial configuration is the number of nodes in $P$ that do not reside in their initial cluster.
    That is,
    $\dist(P, P_I) := \sum_{j=1}^{\ell} | C_j \setminus I_j |$. 
In other words,
at least $\dist(P, P_I)/2$ node swaps are required in order to reach the configuration $P$ from $P_I$, and thus
$\OPT \geq \Delta:= \dist(P_F, P_I) $.
 With each re-partitioning,
  \PPL moves to a configuration that minimizes the distance to the initial configuration $P_I$.
As a result,
\PPL never ends up in a configuration that is more than $\Delta$ (single-node) migrations away from $P_I$.
This invariant ensures that \PPL does not pay too much while recovering $P_F$.

\begin{algorithm}
    \renewcommand{\algorithmicrequire}{\textbf{Input:}}
    \renewcommand{\algorithmicensure}{\textbf{Output:}}
    \begin{algorithmic}
%        \Require 
%        $k, \ell$,
%        initial configuration $P_I$,
%        sequence of  requests $\sigma_1, \dots, \sigma_N$ 
%        \Ensure A final configuration $P_F$ 
        \STATE {For each node $v$ create a singleton component $C_v$ and add it to $\mathcal{C}$}
        \STATE{$P_0 := P_I$}
         \label{line:initcomponents}
        \FOR {each  request $\sigma_t=\{u,v\}, 1 \leq t \leq N$}
        \STATE Let $C_1 \ni u$ and $C_2 \ni v$ be the container components
        \IF{$C_1 \neq C_2$}
        \STATE {Unite the two components into a single component $C'$ and
        $\mathcal{C} = (\mathcal{C}\setminus\set{C_1, C_2}) \cup ~\set{C'}$} \label{line:mergecomponents}
        \IF{$\mathit{cluster}(C_1, P_{t-1}) \neq \mathit{cluster}(C_2, P_{t-1})$
	     \COMMENT{i.e.~if not in the same cluster}    
    	}       
        \STATE {$P_{t} = \mathit{re\mhyphen partition}(P_{t-1}, P_I, \mathcal{C})$} 
         \COMMENT{move to $P$ closest to $P_I$}
        \label{line:rebalance} 
        \ENDIF
        \ENDIF
        \ENDFOR
    \end{algorithmic}
    \caption{Perfect-Partition Learner (\PPL)}
    \label{alg:ppl}
      \end{algorithm}
  
%      \maciek{``re-partition'' procedure name should indicate the fact that this is a specific repartition that is close to initial partition}
      We note that the $\mathit{re\mhyphen partition}$ at Line \ref{line:rebalance} replaces the current configuration $P$ with a perfect-partition closest to $P_I$.
Hence it never moves to a configuration beyond distance $\Delta$.      
\begin{property} \label{prop:dist<OPT}
    Let $P$ be any configuration chosen by \PPL at Line $\ref{line:rebalance}$.
    Then, $\dist(P,P_I) \leq \Delta$.
\end{property}

\begin{lemma}	\label{lemma:rebalancecost}
    The cost of re-partitioning at Line \ref{line:rebalance} is at most $2\cdot\OPT$.
\end{lemma}
\begin{proof}
    Consider the re-partitioning that transforms $P_{t-1}$ to $P_t$ upon the request $\sigma_t$.
    Let $M \subset V$ denote the set of nodes that migrate during this process.
	Let $M^-$ and $M^+$ denote the subset of nodes that (respectively)
    enter or leave their original cluster during the re-partitioning.    
    Then,
    $M = M^+ \cup M^-$.
    Since $|M^-|$ nodes are not in their original cluster before the re-partitioning (i.e., in $P_{t-1}$),
    the distance before the re-partitioning is $\dist(P_{t-1},P_I) \geq | M^-|$.
    Analogously,
     the distance afterwards is $\dist(P_{t},P_I) \geq | M^+|$.
    Thus,
    $|M| \leq \dist(P_{t-1},P_I) + \dist(P_{t},P_I)$.
    By Property \ref{prop:dist<OPT},
    $\dist(P_{t-1},P_I) , \dist(P_{t},P_I) \leq \Delta \leq \OPT$
    and thereby we have	
    $|M| \leq 2\cdot\OPT$.
\end{proof}

\begin{theorem}	\label{thm:upperbound}
    \PPL reaches the final configuration $P_F$ and it is $(2\cdot k\cdot\ell)$-competitive.
\end{theorem}
\begin{proof}
      On each inter-cluster request,
     the algorithm enumerates all $\ell$-way partitions of components
     that are in the same (closest) distance of $P_I$.
     That is, 
     once it reaches a configuration $P$ at distance $\Delta = \dist(P, P_I)$,
     it does not move to a configuration
     $P', \dist(P', P_I) > \Delta$,
     before it enumerates all configurations at distance $\Delta$.
     Therefore,
     \PPL eventually reaches $\Delta=\OPT$ and the configuration $P_F$.
%    including the request that completes revealing of all components that are collocated in $P_F$.
    There are at most $(k-1)\cdot\ell < k\cdot\ell $ calls   to $\mathit{re\mhyphen partition}$
     (i.e., the number of internal edges in $P_F$).
    By Lemma \ref{lemma:rebalancecost},
    each re-partition costs at most $2\cdot\OPT$.
    The total cost is therefore at most $2\cdot\OPT\cdot k\cdot\ell$, which implies the competitive ratio.
%    \mahmoud{This is the cost of moving and the cost of remote comm. is not counted.
%    	So it is 4-competitive (?)}
 \end{proof}

 In Section~\ref{sec:lowerbound} we constructed a $\Omega(k \cdot \ell)$ for \OBRP{}.
 Note that the lower bound holds also in the perfect-partition model, as the constructed input sequence allows \OPT to move to a perfect-partition.
 The corollary is that PPL is optimal.
 

% \section{Corollary and Future Directions}

% The gap between upper and lower bound is still substantial, as the best known algorithm is the component-respecting algorithm of   \cite{repartition-disc} with competitive ratio $O(k^2\cdot\ell^2)$.


\bibliographystyle{ACM-Reference-Format}
\bibliography{references}  

\begin{appendix}

\section{Ommited proofs}

\section{Todos}

\todo{Check naming consistency inside sections, and across them. Possibly contain some definitions in preliminaries.}

\todo{ACM format for review? Is there any specific format like this? With line numbers etc.}

\todo{Include e-mails? Or shall we skip it for now.}

\todo{Check ACM package whitelist}

\todo{Replace package algorithmpseudocode, it is not whitelisted https://www.acm.org/publications/taps/whitelist-of-latex-packages}

\todo{Make an ACM account for the submission}


\todo{CCS concepts are mandatory}

\todo{Spellcheck, grammar (grammarly?)}

\todo{Make sure our manuscript is autor anonymous}

\todo{Line endings (add tildes where necessary)}

\maciek{Consistency: collocated at OR collocated in OR collocated on}

\maciek{Component-respecting vs component respecting. In the same way "a lower bound" and "to lower-bound"}

\end{appendix}

\end{document}
