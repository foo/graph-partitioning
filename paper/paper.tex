 
\documentclass[manuscript,screen=true]{acmart}

\usepackage[utf8]{inputenc}
\usepackage{xspace}
\usepackage{balance}


\begin{document}

\title{Brief Announcement: An Improved Lower Bound for Dynamic Graph Partitioning}


\author{Maciej Pacut}
%\orcid{0000-0002-6379-1490}
\affiliation{%
  \department{Faculty of Computer Science}
  \institution{University of Vienna}
  \country{Austria}
}

\author{Mahmoud Parham} 
\affiliation{%
  \department{Faculty of Computer Science}
  \institution{University of Vienna}
  \country{Austria}
}

\author{Stefan Schmid} 
\affiliation{%
  \department{Faculty of Computer Science}
  \institution{University of Vienna}
  \country{Austria}
}

\copyrightyear{2020} 
\acmYear{2020} 
\setcopyright{acmlicensed}
\acmConference{PODC '20}{July 29-August 2, 2020}{Toronto, Canada}


\newcommand{\todo}[1]{\noindent\color{brown}{todo: #1}\color{black}}


\begin{abstract}
    Lorem
\end{abstract}
    
\maketitle
    
\renewcommand{\shortauthors}{M.~Pacut, M.~Parham, S.~Schmid}

\section{Related Work}

In this paper we study an online balanced partitioning problem.
The static offline version of~the~problem, i.e., a problem variant where
migration is not allowed, where all requests are known in advance, and where
the goal is to find an assignment of $n$ nodes to $\ell$~physical machines, each of~capacity $n/\ell$, is known as the
\emph{$\ell$-balanced graph partitioning problem}. The problem is 
NP-complete, and cannot even be approximated within any finite factor unless P
= NP~\cite{AndRae06}.  The static
variant where $\ell = 2$ corresponds to the minimum bisection problem, which
is already NP-hard~\cite{GaJoSt76}, and 
the currently best approximation ratio is $O(\log n)$~\cite{SarVaz95,ArKaKa99,FeKrNi00,FeiKra02,KraFei06,Raec08}.
The inapproximability of the static variant for general values of $\ell$
motivated research on the bicriteria variant, which can be seen as the offline
counterpart of our capacity augmentation approach. Here, the~goal
is~to~compute a~graph partitioning into $\ell$ components of~size at most~$k$ (where $k > n/\ell$) and the cost of the cut is compared to the optimal (non-augmented)
solution where all components are of a~size at most $n/k$. The variant where
$n \geq 2 \cdot k \cdot \ell$ was considered in
\cite{LeMaTr90,SimTen97,EvNaRS00,EvNaRS99,KrNaSc09}. So far, the~best result~is~an~$O(\!\sqrt{\log n \cdot \log \ell})$-approximation algorithm~\cite{KrNaSc09}.

Our dynamic model is related to online
caching~\cite{SleTar85,FKLMSY91,McGSle91,AcChNo00}, sometimes also referred to
as online caching, where requests for data items (nodes) arrive over time and
need to be served from a cache of finite capacity, and where the number of
cache misses must be minimized. Classic problem variants usually boil down to
finding a smart eviction strategy, such as Least Recently Used (LRU)~\cite{SleTar85}. In our
setting, requests can be served remotely (i.e.,~without fetching the
corresponding nodes to a single physical machine). In this light, our model is more
reminiscent of caching models \emph{with
bypassing}~\cite{EpImLN11,EpImLN15,Irani02}. As a~side result, we show that our problem is
capable of emulating online caching.
A major difference between  these problems is that in the caching problems, each request involves a~single element of the universe, while in our model \emph{both} endpoints of a communication request are subject to~optimization.


\section{Todos}

\todo{Make ORCID visible in ACM style}


\bibliographystyle{ACM-Reference-Format}
\bibliography{references}  

    

\end{document}
