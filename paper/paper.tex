 
\documentclass[manuscript,screen=true]{acmart}

\usepackage[utf8]{inputenc}
\usepackage{xspace}
\usepackage{balance}


\begin{document}

\title{Brief Announcement: An Improved Lower Bound for Dynamic Graph Partitioning}


\author{Maciej Pacut}
%\orcid{0000-0002-6379-1490}
\affiliation{%
  \department{Faculty of Computer Science}
  \institution{University of Vienna}
  \country{Austria}
}

\author{Mahmoud Parham} 
\affiliation{%
  \department{Faculty of Computer Science}
  \institution{University of Vienna}
  \country{Austria}
}

\author{Stefan Schmid} 
\affiliation{%
  \department{Faculty of Computer Science}
  \institution{University of Vienna}
  \country{Austria}
}

\copyrightyear{2020} 
\acmYear{2020} 
\setcopyright{acmlicensed}
\acmConference{PODC '20}{July 29-August 2, 2020}{Toronto, Canada}


\newcommand{\todo}[1]{\noindent\color{brown}{todo: #1}\color{black}}


\begin{abstract}
    Lorem ipsum dolor sit amet, consectetur adipiscing elit, sed do eiusmod tempor incididunt ut labore et dolore magna aliqua. Ut enim ad minim veniam, quis nostrud exercitation ullamco laboris nisi ut aliquip ex ea commodo consequat. Duis aute irure dolor in reprehenderit in voluptate velit esse cillum dolore eu fugiat nulla pariatur. Excepteur sint occaecat cupidatat non proident, sunt in culpa qui officia deserunt mollit anim id est laborum.
\end{abstract}
    
\maketitle
    
\renewcommand{\shortauthors}{M.~Pacut, M.~Parham, S.~Schmid}

\section{Related Work}

In this paper we study an online balanced partitioning problem.
The static offline version of~the~problem, i.e., a problem variant where
migration is not allowed, where all requests are known in advance, and where
the goal is to find an assignment of $n$ nodes to $\ell$~physical machines, each of~capacity $n/\ell$, is known as the
\emph{$\ell$-balanced graph partitioning problem}. The problem is 
NP-complete, and cannot even be approximated within any finite factor unless P
= NP~\cite{AndRae06}.  The static
variant where $\ell = 2$ corresponds to the minimum bisection problem, which
is already NP-hard~\cite{GaJoSt76}, and 
the currently best approximation ratio is $O(\log n)$~\cite{SarVaz95,ArKaKa99,FeKrNi00,FeiKra02,KraFei06,Raec08}.
The inapproximability of the static variant for general values of $\ell$
motivated research on the bicriteria variant, which can be seen as the offline
counterpart of our capacity augmentation approach. Here, the~goal
is~to~compute a~graph partitioning into $\ell$ components of~size at most~$k$ (where $k > n/\ell$) and the cost of the cut is compared to the optimal (non-augmented)
solution where all components are of a~size at most $n/k$. The variant where
$n \geq 2 \cdot k \cdot \ell$ was considered in
\cite{LeMaTr90,SimTen97,EvNaRS00,EvNaRS99,KrNaSc09}. So far, the~best result~is~an~$O(\!\sqrt{\log n \cdot \log \ell})$-approximation algorithm~\cite{KrNaSc09}.

Our model is related to online
caching~\cite{SleTar85,FKLMSY91,McGSle91,AcChNo00}, sometimes also referred to
as online caching, where requests for data items (nodes) arrive over time and
need to be served from a cache of finite capacity, and where the number of
cache misses must be minimized. Classic problem variants usually boil down to
finding a smart eviction strategy, such as Least Recently Used (LRU)~\cite{SleTar85}. In our
setting, requests can be served remotely (i.e.,~without fetching the
corresponding nodes to a single physical machine). In this light, our model is more
reminiscent of caching models \emph{with
bypassing}~\cite{EpImLN11,EpImLN15,Irani02}. As a~side result, we show that our problem is
capable of emulating online caching.
A major difference between  these problems is that in the caching problems, each request involves a~single element of the universe, while in our model \emph{both} endpoints of a communication request are subject to~optimization.

\section{Practical Motivation}


Distributed cloud applications, including batch processing
applications such as MapReduce, streaming applications such as Apache Flink or
Apache Spark, and scale-out databases and key-value stores such as Cassandra,
generate a~significant amount of network traffic and a~considerable fraction
of their runtime is due to network activity~\cite{MogPop12}. For example,
traces of jobs from a Facebook data center reveal that network transfers on
average account for 33\% of the execution time~\cite{orchestra}. In such
applications, it is desirable that frequently communicating virtual machines
are \emph{collocated}, i.e., mapped to the same physical server: 
communication across the network (i.e., inter-server communication) induces
network load and latency. However, migrating virtual machines between servers
also comes at a price: the state transfer is bandwidth intensive, and may even
lead to short service interruptions. Therefore the goal is to design online
algorithms that find a good trade-off between the inter-server communication
cost and the migration cost.


\vspace{1cm}

We study virtual network embeddings in the scenario where virtual machines can be migrated during runtime to another physical machine.
The possibility of migration allows reacting to unpredictable communication patterns.
For example, if some distant nodes communicate often, it is vital to reduce their distance to save network bandwidth.
The objective is to~minimize the total network bandwidth used for communication and for migration.


We assume that the communication patterns are not known in advance to our algorithm.
We measure the~quality of~presented algorithmic solutions by competitive analysis~\cite{borodin-book}, which is well-suited for problems that are online by their nature.
In the competitive analysis, the goal is to~optimize \emph{the competitive ratio} of a given online algorithm: the ratio of its cost to the cost of~an~optimal offline algorithm that knows the whole input sequence in advance.

In the dynamic scenario, we assume that the physical substrate network is a~tree of height one.
That is, every physical machine (leaf) is connected directly to the root (that has no hosting capabilities).
A single physical machine hosts a fixed number of virtual machines.
The model restricted to such networks becomes a variant of online graph clustering.
That is, we are given a~set of~$n$ nodes (virtual machines) with time-varying pairwise
communication patterns, which have to be partitioned into~$\ell$~physical machines, each of
capacity $k=n/\ell$.

Intuitively, we would like to minimize inter-machine
interactions by mapping frequently communicating nodes to the same physical machine
Since communication patterns change over time, the~nodes should be \emph{repartitioned}, in
an online manner, by \emph{migrating} them between physical machines.
The~objective is to minimize the weighted sum of inter-machine communication and repartitioning costs.
The former is defined as the number of communication requests between nodes placed at distinct physical machines, and the latter as the number of migrations.


The possibility to perform a migration uncovers algorithmic challenges:
\begin{itemize}

\item \emph{Serve remotely or migrate?} For a brief communication
pattern, it may not be worthwhile to collocate the nodes: the migration cost might
be too large in comparison to~communication costs.

\item \emph{Where to migrate, and what?}
If an algorithm decides to collocate nodes $x$ and~$y$, the~question becomes
how. Should $x$ be migrated to the physical machine holding $y$, $y$ to the one holding
$x$, or should both nodes be migrated to a new machine?

\item \emph{Which nodes to evict?}
The space of the desired destination physical machine may not be sufficient. In
this case, the~algorithm needs to decide which nodes to ``evict'' (migrate to
other machines), to free up space.

\end{itemize}


\section{Problem Definition}


Formally, the online \emph{Balanced RePartitioning} problem (BRP) is defined as
follows. There is a set of $n$ nodes, initially distributed arbitrarily
across $\ell$~clusters, each of size~$k$. We call two nodes~$u,v\in V$
\emph{collocated} if they are in the same cluster.

An input to the problem is a sequence of communication requests $\sigma =
(u_1,v_1),$ $(u_2,v_2),$ $(u_3,v_3), \ldots$, where pair $(u_t,v_t)$ means that
the nodes $u_t,v_t$ exchange a fixed amount of data.  At any time~$t$, an online algorithm needs to serve the~communication
request~$(u_t,v_t)$. Right before serving the request, the online algorithm
can repartition the nodes into new clusters. We assume that
a~communication request between two collocated nodes costs 0. The cost of a~communication request between two nodes located in different clusters is
normalized to~1, and the cost of migrating a node from one cluster to another
is~$\alpha \geq 1$, where $\alpha$ is a parameter (an~integer). For any
algorithm $ALG$, we denote its total cost (consisting of communication plus
migration costs) on sequence $\sigma$ by $ALG(\sigma)$.

\section{Todos}

\todo{Make ORCID visible in ACM style}


\bibliographystyle{ACM-Reference-Format}
\bibliography{references}  

    

\end{document}
