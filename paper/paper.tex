 
\documentclass[manuscript,screen=true]{acmart}

\usepackage[utf8]{inputenc}
\usepackage{xspace}
\usepackage{balance}
\usepackage{amsmath,amsfonts,mathtools,amsthm}
\usepackage{algorithmic}
\usepackage{algorithm}

\mathchardef\mhyphen="2D

\title{Brief Announcement: An Improved Lower Bound for Dynamic Balanced Graph Partitioning}


\author{Maciej Pacut}
\email{maciek.pacut@gmail.com}
\orcid{0000-0002-6379-1490}
\affiliation{%
  \institution{Faculty of Computer Science, University of Vienna}
  \country{Austria}
}


\author{Mahmoud Parham} 
\email{mahmoud.parham@univie.ac.at}
\orcid{0000-0002-6211-077X}
\affiliation{%
  \institution{Faculty of Computer Science, University of Vienna}
  \country{Austria}
}

\author{Stefan Schmid} 
\email{stefan_schmid@univie.ac.at}
\affiliation{%
  \institution{Faculty of Computer Science, University of Vienna}
  \country{Austria}
}

\copyrightyear{2020} 
\acmYear{2020} 
\setcopyright{acmlicensed}
\acmConference{PODC '20}{July 29-August 2, 2020}{Toronto, Canada}

\keywords{online algorithms, competitive analysis}


%%%%%%%%%%%%%%%%%%%%%%%%%%%%%%%%%%%%%%%%%%%%%%%%&&
%%%%%%%%%%%%%%%%%%%%%%%%%%%%%%%%%%%%%%%%%%%%%%%%&&
%  our macros start
%%%%%%%%%%%%%%%%%%%%%%%%%%%%%%%%%%%%%%%%%%%%%%%%&&
%%%%%%%%%%%%%%%%%%%%%%%%%%%%%%%%%%%%%%%%%%%%%%%%&&

\newcommand{\OPT}{\mathit{OPT}}
\newcommand{\ALG}{ALG}
\newcommand{\OBRP}{BRP}
\newcommand{\PPOBRP}{PP-BRP}
\newcommand{\TAlg}{{\ensuremath{\textsf{ALG}_{3}}}\xspace} % we should change this name



\newtheorem{claim}{Claim}
\newtheorem{fact}{Fact}
\newtheorem{rem}{Remark}
\newtheorem{observation}{Observation}
\newtheorem{property}{Property}


\DeclarePairedDelimiter\pair{(}{)}
\DeclarePairedDelimiter\set{\{}{\}}

\DeclarePairedDelimiter{\ceil}{\lceil}{\rceil}
\DeclarePairedDelimiter{\floor}{\lfloor}{\rfloor}

\newcommand\mahmoud[1]{\color{green}\textbf{\\ Mahmoud: #1}\\\color{black}}
\newcommand\stefan[1]{\color{blue}\textbf{\\ Stefan: #1}\color{black}}
\newcommand\maciek[1]{\color{brown}\textbf{\\ Maciek: #1}\color{black}}


\newcommand{\todo}[1]{\noindent\color{brown}{todo: #1}\color{black}}


\begin{document}


\begin{abstract}
  In the online Graph RePartitioning problem,
  a sequence of communication requests arrives online,
  between pairs of $n$ nodes that are partitioned into $\ell$ clusters
  each of size $k$ (i.e., $n = \ell \cdot k$).
  The objective is to serve these requests while paying a cost that is reasonably close to the optimal cost of serving them given  the whole sequence in advance. 
  This is achieved by opting a sequence of  partitions (a.k.a.~configurations)
  such that frequently communicating nodes (from different clusters)
  are relocated to the same cluster before they inflict too much communication cost.
  A \emph{reconfiguration} (i.e., re-partitioning) is performed by
  moving (i.e., migrating) nodes between clusters at a cost.
  The goal is to jointly minimize communication and migration costs while serving the sequence.

	The best known deterministic algorithm is $O(k^2\cdot \ell^2)$-competitive \cite{?????} \maciek{We should cite \cite{repartition-disc} here.},
	and the best known lower bound for the competitive ratio of
	deterministic algorithms is $\Omega(k)$,
	which uses a reduction from the online paging problem.
	  In this paper, we provide the lower bound of $\Omega(k\cdot \ell)$.
    We also present an optimal algorithm for the case of $k=3$.
  Finally,
  we provide a $O(k\cdot \ell)$-competitive algorithm for the variant
  where the optimal configuration is always a fixed partition
  that incurs zero communication cost through the whole request sequence.
\end{abstract}
    
\maketitle
    
\renewcommand{\shortauthors}{M.~Pacut, M.~Parham, S.~Schmid}

\section{Introduction}

Main technical contribution of this paper is the lower bound $\Omega(k\cdot\ell)$ for the competitive ratio of any deterministic algorithm for Online Balanced Partition problem (cf. Section~\ref{sec:lowerbound}).
The best known lower bound so far was $\Omega(k)$ \cite{repartition-disc}
which involves only two cluster.


So far, the competitive ratio of the best known algorithm was the component-based algorithm with the ratio $O(k^2\cdot \ell^2)$ \cite{repartition-disc}.
We present an optimal algorithm for $k=3$ in Section~\ref{sec:k3}.
In Section~\ref{sec:ppl}, we present a $O(k\cdot \ell)$-competitive algorithm for the restricted scenario
where the communication graph (i.e., request sequence) admits a \emph{perfect partition} (i.e., a multi-way cut of cost zero) that is also an optimal configuration for the offline problem.
For this setting, the algorithm is optimal, as the lower bound utilized an input sequence that has a perfect partition.

\section{Related Work}

The problem was studied under resource augmentation \cite{repartition-disc}, and under stochastic assumptions of input\maciek{cite}.\maciek{cite other papers that cite DISC?}

%The model is related to online
%paging~\cite{SleTar85,FKLMSY91,McGSle91,AcChNo00}.
In our
setting,
each  requests between two nodes  not in the same cluster
(i.e., inter-cluster requests) inflicts a cost until the nodes are placed in the same cluster.
This is a reminiscent of caching models \emph{with
rejection} penalty~\cite{EpImLN11,EpImLN15,Irani02}.

The static offline version of~the~problem, called the
\emph{$\ell$-balanced graph partitioning problem} is 
NP-complete, and cannot even be approximated within any finite factor unless P
= NP~\cite{AndRae06}. The static
variant where $\ell = 2$ corresponds to the minimum bisection problem, which
is already NP-hard~\cite{GaJoSt76}, and 
the currently best approximation ratio is $O(\log n)$~\cite{SarVaz95,ArKaKa99,FeKrNi00,FeiKra02,KraFei06,Raec08}.




\section{Problem Definition}
\label{sec:problem-definition}

Formally, the online \emph{Balanced RePartitioning} problem (\OBRP{}) is defined as
follows. We are given a set of $n$ nodes,
initially arbitrarily partitioned into $\ell$~clusters,
each of size~$k$.
We say that nodes in $C \subset V$ are \emph{collocated}
% in a partition $P$
if they are assigned to the same cluster.
%\maciek{We can ommit the formal definition: "I.e.,
%$\forall u,v \in C: \mathit{cluster} (u,P) = \mathit{cluster} (v,P)$.". %However, if we use the notion of cluster later, then TODO: define cluster}
%\maciek{We can ommit the formal definition: "I.e.,
%$\forall u,v \in C: \mathit{cluster} (u,P) = \mathit{cluster} (v,P)$.". However, if we use the notion of cluster later, then TODO: define cluster}
The input consists of a sequence of pairwise communication requests
$\sigma = (u_1,v_1),$ $(u_2,v_2),$ $(u_3,v_3), \ldots$,
where a pair $(u_t,v_t)$ indicates that nodes $u_t$ and $v_t$ exchange a fixed amount of data.
Before serving the request,
the online algorithm can re-partition,
i.e.,
it may (optionally) move some of the nodes into clusters different than their current assigned clusters.
Once the re-partition is executed,
we serve
the communication request between two  nodes at cost~0 if they are collocated,
or at cost~1 if they are located in different clusters.
The cost of migrating a node from one cluster to another
is~$\alpha \in \mathbb{Z}^+$.
For any algorithm $ALG$,
its cost,
denoted by $ALG(\sigma)$,
is the total cost of communications and
the cost of migrations performed by $ALG$ while serving the sequence $\sigma$.

\subsection{The Perfect Partition Variant}	\label{sec:PP}
In Section \ref{sec:ppl},
we give an algorithm for a restricted variant of  \OBRP{},
where the communication graph can be partitioned into $\ell$ clusters without any inter-cluster edges.
Moreover, we assume an optimal offline algorithm (OPT) moves to this \emph{perfect partition}
at the beginning and stays there permanently.
We refer this variant as \PPOBRP{}.
The task of an online algorithm for \PPOBRP{} is to ``recover" (or learn) the perfect partition while not paying too much more than OPT.
This implies that on a request between any two non-collocated nodes,
any online algorithm must eventually collocate them,
as otherwise it not competitive under a sequence that repeats the same request arbitrarily many times.
Hence,
any competitive algorithm,
must at some point collocate all nodes of a  connected sub-graph of the communication graph.
We refer to this connected sub-graph as a \emph{component}.
An algorithm that  keeps all components collocated is \emph{component-respecting}.

\section{Lower bound} %$\Omega(k\cdot \ell)$ for Competitive Ratio of Any Deterministic Algorithm}
\label{sec:lowerbound}


\maciek{Here, we would like to introduce insights about the proof, emphasize the conditions for the bound to work (such as $k\geq 3$), and maybe compare to previously known lower bounds. Also, introduce the notation and definitions that are needed.}
\maciek{We should describe what we mean by  components (that we force alg to keep on the same server). We use components differently in $k=3$, but we denote them by components, too. This is understandable, but must be emphasized.}

\maciek{This paragraph should be included in the corollary from PPL, right after its competitive proof.}
In this section we show a lower bound of $\Omega(k \cdot \ell)$ for \OBRP{}.
The constructed input sequence allows OPT to move to a perfect partition, and hence
the construction also constitutes a lower bound for \PPOBRP{}.
This complements an upper bound of $O(k \cdot \ell)$
for \PPOBRP{} that we present in Section \ref{sec:ppl}.

\begin{theorem}
	The competitive ratio of any deterministic algorithm for \OBRP{} is in $\Omega(k\cdot \ell)$ for any $k\geq 3$.
\end{theorem}

\begin{proof}
	We construct an instance of the problem with $\ell$ clusters 
	$\set{ S_1, S_2,\dots , S_{\ell}}, |S_i|  = k$.
	Let $I(C)$ denote the cluster where nodes in a component $C$ are located initially.
	%	\maciek{So far components were not introduced; the best place to introduce these would be around the definition of \PPOBRP}.
	Fix any online algorithm \ALG{}.
	We construct the input sequence for \ALG{} as follows.
	First,
	we issue $k-2$ (internal) requests so that \ALG{} can form a component of $k-1$
	nodes on the cluster $S_1$.
	Recall that internal requests does not incur any cost for \ALG{}.
	If \ALG{} at any point splits a component
	(i.e., spreads its nodes over two or more clusters),
	then we continue to issue requests between non-collocated nodes of the component until \ALG{} collocates the nodes of the component.
	Hence, if \ALG{} never collocates them then it is not competitive.
  
  \maciek{Components are not introduced. Thus, the reader does not know what a single node means.}
	Let $x_0$  be the only single node left on $S_1$ and  $y_0 \in S_2$ be any single node on $S_2$.
	Next,
	we issue a request between $x_0$ and $y_0$.
	Since this is an external \maciek{external request is not defined at this point} request,
	\ALG{} joins them into one component and collocates them in some cluster other than $S_1$.
	For this,
	\ALG{} moves to a new configuration
	that replaces $x_0$ and $y_0$ with two other single nodes $x_1$ and $y_1$ respectively.
	
	Next,
	we issue a request between $x_1$ and the largest component $C$ s.t.~$I(C) = I(x_1)$.
	\ALG{} must collocate $x_1$ and $C$ on some cluster other than $S_1$ and
	consequently replaces $x_1$ with some other (single) node $x_2$.
	We repeat issuing requests between the single node $x_i$ on $S_1$ and the largest component $C'$ s.t.~$I(C')=I(x_i)$,
  until there are only two single nodes left that  originate from the same cluster.
	Formally, we denote these two single nodes by $x^*, y^*$, and we have $I(x^*) = I(y^*)$, and~for any other pair of single nodes
	$x'$ and $y'$,
	we have $I(x') \neq I(y')$.
	At this point there are at most $\ell+1$ single nodes left,
	otherwise there would be more pairs of single nodes that were initially in the same cluster.
	
	Given this sequence of requests,
	the optimal strategy is to migrate $\set{x_0,y_0}$ to the cluster $I(x^*)$ by
	swapping $\set{x_0,y_0}$ with $\set{x^*,y^*}$.
	Hence,
	OPT pays for $4$ node migrations and
	\ALG{} incurs at least one migration for each node in the sequence $X := x_0, x_1,\dots$.
	We exclude at most $(k-1) + ( \ell+1)$ nodes out of $k \cdot \ell$ nodes,
	therefore $|X| \geq k \cdot \ell - k - \ell \in \Omega(k\cdot\ell)$.
\end{proof}


\section{An Optimal Algorithm for $k=3$}

In this section we show that the classic \emph{rent-or-buy} approach\cite{borodin-book}\maciek{any better source? ski rental?} allows to obtain an optimal algorithm for $k=3$.
Upon receiving $\alpha$ requests between a pair of nodes, we collocate them in one server until the end of an epoch (that we define precisely later).
To this end, we maintain \emph{components of nodes}, in similar fashion to previous approaches\cite{repartition-disc}.

%We say that the request $(u,v)$ is \emph{external}, if the algorithm is in the configuration, where $u$ and $v$ are located at different servers.

%We define the algorithm \TAlg in the following way.
Now we describe the algorithm \TAlg.
\TAlg partition nodes into components, and
intially each node belongs to its own singleton component.
For each pair of nodes \TAlg maintains a counter, initiated $0$. 
Upon receiving a request to a pair that is not collocated at one server, it increases the counter by $1$.
If the counter for a pair $(u,v)$ reaches $\alpha$, \TAlg merges the components of $u$ and $v$, and moves to the closest partition, in which each component has all of its nodes collocated in one server.
If no such partition exists, \TAlg resets all components to singletons and all counters to $0$ and ends the phase.

%\maciek{Shouldn't we think about whole component cuts instead of single edge? Hmm. NO!}

We say that \TAlg is \emph{component respecting}, i.e., it maintains the invariant that nodes of each component are located at the same server.
Observe that this implies that \TAlg never increases any counter above $\alpha$.

In our analysis, we distinguish among three types of clusters: $C_1, C_2, C_3$. In a cluster of type $C_i$, the size of the largest component contained in this cluster is $i$.

\begin{lemma}
  \label{lem:1req}
  The cost of migrations during a single repartition of \TAlg is at most $3\cdot\alpha$.
\end{lemma}

\begin{proof}
  If no partition of \TAlg's components exist, then it performs no repartition and the cost is $0$.
  Hence, in the following we bound the cost of moving to a component-respecting partition.

  %We distinguish between three types of clusters: $C_1, C_2, C_3$,
  %which we define as follows.
  %A cluster of type $C_1$ the cluster contains $3$ singleton components (this is also the initial configuration of any cluster).
  %A cluster of type $C_2$  contains one component of size $2$ and one component of size $1$.
  %Finally, a cluster of type $C_3$  contains one component of size $3$.
  
    Consider a request $(u, v)$ that triggered the repartitiong and let $U, V$ be clusters of these nodes.
    Note that $U\neq V$, as such a request cannot trigger a reconfiguration.
    We consider cases upon types of clusters $U$ and $V$.
    Note that this is impossible that either $U$ or $V$ is of type $C_3$, as otherwise we encounter a component of size $4$ and this would contradict the case assumption that the component-respecting partition exists.
    If either $U$ or $V$ is of type $C_1$, then this cluster can fit the merged component, and the rebalance is local within $U$ and $V$, for the cost of at most $3$ migrations.
  
    Finally, we consider the case where both $U$ and $V$ are of type $C_2$. Note that $(u,v)$ cannot both belong to a component of size $2$, as this would mean that \TAlg has the component of size $4$, a contradiction with the case assumption that the component-respecting partition exists. 
    If either $u$ or $v$ belongs to a component of size $2$, then it suffices to exchange components of size $1$ between $U$ and $V$.
    Finally, if $u$ and $v$ belong to components of size $1$, then we must place them in a cluster different from $U$ and $V$.
    Note that in such case, a $C_1$-type cluster $W$ exists, as otherwise no component-respecting partition exists. In this case \TAlg exchanges the nodes $u$ and $v$ with any two nodes of $W$, for the total cost of $2\cdot \alpha$.
\end{proof}



\begin{theorem}
  \TAlg is $O(\ell)$-competitive.
\end{theorem}
\begin{proof}
  Fix a completed phase, and consider the state of \TAlg's counters at the end of it.
  We consider the incomplete phase later in this proof.
  We say that the pair $(u, v)$ is \emph{saturated}, if the counter has value $\alpha$, and \emph{unsaturated} otherwise.

  \maciek{Rewrite the paragraph to emphasize in the definition that requests that did not increase the counter are not included}
  We distinguish two disjoint subsequences of input $\sigma$ that arrived during the phase: $\sigma_I$ and $\sigma_E$, called \emph{internal} and \emph{external}, respectively.
  We only account for the requests to nodes that were not collocated by \TAlg at the moment of their arrival, i.e., the requests that increased a counter.
  The subsequence $\sigma_E$ contains requests to pairs that belong to the same component of \TAlg{} at the end of the phase, and $\sigma_I$ denotes the requests to pairs that did not belong to the same component of \TAlg.

  %Let $A^I$ be the cost of (extra-cluster) communication incurred in this phase by \TAlg between pairs that belonged to the single component at the end of the phase.
  %Let $A^E$ be the cost of (extra-cluster) communication incurred in this phase between the nodes that belong to different components at the end of this phase.
  Let $\TAlg(M)$ be the cost of migrations performed by \TAlg in this phase.
  During a single phase, it performs at most $2\cdot \ell$ component merge operations, as
  exceeding this number means that a component of size $4$ exists.
  Combining this with Lemma~\ref{lem:1req} gives us $\TAlg(M) \leq 6\cdot\alpha\cdot\ell$.
  %Together with Lemma~\ref{lem:1req}, this allows us to bound the cost of migrations, $\TAlg(M) \leq 6\cdot\alpha\cdot\ell$.
  
  Now we bound $\TAlg(\sigma_I)$.
  Cluster of type $C_3$ contributes at most $(3\cdot \alpha - 1)$ to $\TAlg(\sigma_I)$, as at most $2$ of pairs of the component are saturated and contribute $\alpha$ each, and the remaining unsaturated pair contributes at most $\alpha-1$.
  Other cluster types contribute less: $C_1$ contributes $0$ and $C_2$ contributes $\alpha$.
  Summing this over all $\ell$ clusters gives us $\TAlg(\sigma_I) \leq \ell \cdot (3\cdot \alpha-1)$.

  %We bound $A^E$ by $k^2 \cdot (\alpha - 1)$, as no more than $k^2$ pairs are unsaturated, and each of them contributes at most $\alpha -1$.
  %\maciek{not needed most likely}

  In total, the cost of \TAlg is at most $\TAlg(\sigma_I) + \TAlg(\sigma_E) + \TAlg(M) \leq 9\alpha\cdot \ell + \TAlg(\sigma_E)$ during this phase.

  \medskip

  Now we lower-bound the cost of $\OPT$.
  By $\OPT(\sigma_I)$ and $\OPT(\sigma_E)$ we denote the cost of $\OPT$ on these input sequences ($\sigma_I$ and $\sigma_E$ are defined with respect to components of \TAlg in this phase).
  By $\OPT(M)$ we denote the cost of migrations performed by $\OPT$ in this phase.
  
  We split the cost of $\OPT$ into parts coming from serving $\sigma_I$ and $\sigma_E$.
  While serving these requests, $\OPT$ may perform migrations, and we account for them in both parts: we separately bound $\OPT$ by $\OPT(\sigma_I) + \OPT(M)$ and $\OPT(\sigma_E) + \OPT(M)$.
  Combining those bounds gives us $\OPT \geq \max\{\OPT(\sigma_I) + \OPT(M), \OPT(\sigma_E) + \OPT(M)\} \geq (\OPT(\sigma_I) + \OPT(M)) / 2 + (\OPT(\sigma_E) + \OPT(M)) / 2$.

  %Let $O^M$ be the cost of migrations performed by $\OPT$ during the phase.
  %Let $O^I$ be the cost of serving requests between nodes that were put in one component by \TAlg during this phase.
  %Let $O^E$ be the cost serving requests between nodes that \TAlg did not put in the same component during that phase.

  %First, we estimate the cost related to $\sigma_I$.
  We have $OPT(M) + OPT(\sigma_I) \geq \alpha$, as the components of \TAlg{} could not be partitioned without splitting them, and there exists a pair of nodes with at least $\alpha$ requests between them.
  $\OPT$ either served them remotely or performed a migration.

  \medskip
  \maciek{The weakest paragraph}
  \maciek{x must be a sequence of requests, not the number; better name for x}
  Before we bound the competitive ratio, we relate the costs of $\TAlg$ and $\OPT$ with respect to $\sigma_E$.
  We begin by lower-bounding the number of requests from $\sigma_E$ that $\OPT$ must account (by either serving remotely or migrating), and we denote their number by $x$.
  It holds that $x \geq \TAlg(\sigma_E) - 3\cdot\ell\cdot(\alpha - 1)$, as it received all the requests that \TAlg did, but it could ignore the cost of serving at most $3\cdot\ell$ pairs of nodes, as it might have them collocated in its clusters.
  Each such pair of nodes account for at most $\alpha-1$ requests (the pairs from $\sigma_E$ are unsaturated) in the cost of $\TAlg(\sigma_E)$, hence collocating them would decrease their number by at most $3\cdot\ell\cdot(\alpha - 1)$.

  Now we bound the cost incurred by $\OPT$ while serving requests from $x$.
  Instead of serving all of them remotely, $\OPT$ can perform migrations to decrease its cost.
  By performing a swap of nodes $(u,v)$, $\OPT$ collocates $u$ with two nodes $u', u''$, and $v$ with two nodes $v'$, $v''$.
  This allows to serve requests between $(u,u')$, $(u,u'')$, $(v,v')$ and $(v,v'')$ for free afterwards.
  (express that between a pair of nodes we accounted $\alpha-1$ requests in $x$)
  Hence, by performing a migration for the cost $\alpha$, $\OPT$ does not pay the remote serving costs for at most $4\cdot (\alpha - 1)$ requests from $x$ ($x$ is not a sequence but its length!).
  Finally, for serving $x$, $\OPT$ pays at least $x \cdot \frac{\alpha}{4\cdot (\alpha-1)}$.
  This gives us
  \begin{equation*}
    \OPT(\sigma_E) + \OPT(M)  \geq (\TAlg(\sigma_E)-3\cdot\ell\cdot(\alpha - 1)) \cdot \frac{\alpha}{4\cdot (\alpha-1)}
    \geq \TAlg(\sigma_E) / 4 - 4\cdot \ell \cdot \alpha.
  \end{equation*}
  From that, we obtain $\TAlg(\sigma_E) \geq 4\cdot(\OPT(\sigma_E)+\OPT(M)) + 16\ell \cdot \alpha$.
  With that, we are ready to bound the competitive ratio.
  Let $E := \OPT(\sigma_E) + \OPT(M)$. Finally, using above bounds,
  \begin{equation*}
    \frac{\TAlg(\sigma)}{\OPT(\sigma)} \leq \frac{9\alpha \cdot \ell + \TAlg(\sigma_E)}{(\alpha + \OPT(\sigma_E) + \OPT(M))/2} \leq \frac{50\alpha\cdot\ell + 8\cdot E}{\alpha + E} \leq 50 \ell = O(\ell).
  \end{equation*}

  \medskip

  Now we consider the last, unfinished phase.
  Consider the case, where the unfinished phase is also the first one.
  Then, we cannot charge $\OPT$ due to unability to partition the components.
  Instead, we use the fact that \TAlg and $\OPT$ started with the same initial configuration.
  We charge $\OPT$ $\alpha$ for the first external $\alpha$ requests or a migration,
  and we follow the analysis regarding the unsaturated requests.
  If the input finished before first $\alpha$ external requests, then \TAlg is optimal.
  Now, consider the case, where there are at least two phases, then we split the cost $\alpha$ charged in the penultimate phase into last two phases, and follow the analysis regarding the unsaturated requests.
  This way, the competitive ratio increases at most twofold.
\end{proof}


\section{Recovering a Perfect Partition}	\label{sec:ppl}
%\subsection{Perfect Partition Model}\label{sec:perfectPartition}
%\maciek{Here introduce a restricted model of \PPOBRP{}, the restricted variant where the input sequence can be perfectly partitioned}
%\maciek{ every algorithm must colocate every communicating pair.
%	\mahmoud{updated!}}
%\maciek{Note that the algorithm from Section~\ref{sec:k3} is $O(\ell)$-competitive for the Perfect Partition Model, as during the first phase, a perfect partition of input exists.}

%\subsection{An $O(k\cdot \ell)$-competitive Algorithm for Perfect Partition}
     
We assume OPT begins with the initial configuration
$P_I = I_1, \dots, I_{\ell}$ and moves to the final partition
$P_F = F_1, \dots, F_{\ell}$.
 The \emph{distance} of a configuration $P = C_1, \dots, C_{\ell}$ from the initial configuration is the number of nodes in $P$ that do not reside in their initial cluster.
    That is,
    $\mathit{dist}(P, P_I) := \sum_{j=1}^{\ell} | C_j \setminus I_j |$. 
In other words,
at least $\mathit{dist}(P, P_I)/2$ node swaps are required in order to reach the configuration $P$ from $P_I$, and thus
$\OPT \geq \Delta:= dist(P_F, P_I) $.
 With each re-partitioning\maciek{Use consistently: repartition or re-partition. IDK which one is better.},
  PPL moves to a configuration that minimizes distance to the initial configuration $P_I$.
As a result,
ON never ends up in a configuration that is more than $\Delta$ away from $P_I$.
This invariant ensures that ON does not pay too much while reaching $P_F$.


\begin{algorithm}
    \renewcommand{\algorithmicrequire}{\textbf{Input:}}
    \renewcommand{\algorithmicensure}{\textbf{Output:}}
    \begin{algorithmic}
%        \Require 
%        $k, \ell$,
%        initial configuration $P_I$,
%        sequence of  requests $\sigma_1, \dots, \sigma_N$ 
%        \Ensure A final configuration $P_F$ 
        \STATE {For each node $v$ create a singleton component $C_v$ and add it to $\mathcal{C}$}
        \STATE{$P_0 := P_I$}
         \label{line:initcomponents}
        \FOR {each  request $\sigma_t=\{u,v\}, 1 \leq t \leq N$}
        \STATE Let $C_1 \ni u$ and $C_2 \ni v$ be the container components
        \IF{$C_1 \neq C_2$}
        \STATE {Unite the two components into a single component $C'$ and
        $\mathcal{C} = (\mathcal{C}\setminus\set{C_1, C_2}) \cup ~\set{C'}$} \label{line:mergecomponents}
        \IF{$\mathit{cluster}(C_1, P_{t-1}) \neq \mathit{cluster}(C_2, P_{t-1})$
	     \COMMENT{i.e.~if not in the same cluster}    
    	}       
        \STATE {$P_{t} = \mathit{re\mhyphen partition}(P_{t-1}, P_I, \mathcal{C})$} 
         \COMMENT{move to $P$ closest to $P_I$}
        \label{line:rebalance} 
        \ENDIF
        \ENDIF
        \ENDFOR
    \end{algorithmic}
    \caption{Perfect Partition Learner (PPL)}
    \label{alg:ppl}
      \end{algorithm}
  
%      \maciek{``re-partition'' procedure name should indicate the fact that this is a specific repartition that is close to initial partition}
      We note that the $\mathit{re\mhyphen partition}$ at Line \ref{line:rebalance} replaces the current configuration $P$ with a (perfect) partition closest to $P_I$.
Hence it never moves to a configuration beyond distance $\Delta$.      
\begin{property} \label{prop:dist<OPT}
    Let $P$ be any configuration chosen by Algorithm \ref{alg:ppl} at Line $\ref{line:rebalance}$.
    Then, $\mathit{dist}(P,P_I) \leq \Delta$.
\end{property}

\begin{lemma}	\label{lemma:rebalancecost}
    The cost of re-partitioning at Line \ref{line:rebalance} is at most $2\cdot\OPT$.
\end{lemma}
\begin{proof}
    Consider the re-partitioning that transforms $P_{t-1}$ to $P_t$ upon the request $\sigma_t$.
    Let $M \subset V$ denote the set of nodes that migrate during this process.
	Let $M^-$ and $M^+$ denote the subset of nodes that (respectively)
    enter or leave their original cluster during the re-partitioning.    
    Then,
    $M = M^+ \cup M^-$.
    Since $|M^-|$ nodes are not in their original cluster before the re-partitioning (i.e., in $P_{t-1}$),
    the distance before the re-partitioning is $\mathit{dist}(P_{t-1},P_I) \geq | M^-|$.
    Analogously,
     the distance afterwards is $\mathit{dist}(P_{t},P_I) \geq | M^+|$.
    Thus,
    $|M| \leq \mathit{dist}(P_{t-1},P_I) + \mathit{dist}(P_{t},P_I)$.
    By Property \ref{prop:dist<OPT},
    $\mathit{dist}(P_{t-1},P_I) , \mathit{dist}(P_{t},P_I) \leq \Delta \leq \OPT$
    and thereby we have	
    $|M| \leq 2\cdot\OPT$.
\end{proof}

\begin{theorem}	\label{thm:upperbound}
    PPL reaches the final configuration $P_F$ and it is $(2\cdot k\cdot\ell)$-competitive.
\end{theorem}
\begin{proof}
      On each inter-cluster request,
     the algorithm enumerates all $\ell$-way partitions of components
     that are in the same (closest) distance of $P_I$.
     That is, 
     once it reaches a configuration $P$ at distance $\Delta = \mathit{dist} (P, P_I)$,
     it does not move to a configuration
     $P', \mathit{dist} (P', P_I) > \Delta$,
     before it enumerates all configurations at distance $\Delta$.
     Therefore,
     PPL eventually reaches $\Delta=\OPT$ and the configuration $P_F$.
%    including the request that completes revealing of all components that are collocated in $P_F$.
    There are at most $(k-1)\cdot\ell < k\cdot\ell $ calls   to $\mathit{re\mhyphen partition}$
     (i.e., the number of internal edges in $P_F$).
    By Lemma \ref{lemma:rebalancecost},
    each re-partition costs at most $2\cdot\OPT$.
    The total cost is therefore at most $2\cdot\OPT\cdot k\cdot\ell$, which implies the competitive ratio.
%    \mahmoud{This is the cost of moving and the cost of remote comm. is not counted.
%    	So it is 4-competitive (?)}
 \end{proof}


% \section{Corollary and Future Directions}

% The gap between upper and lower bound is still substantial, as the best known algorithm is the component-respecting algorithm of   \cite{repartition-disc} with competitive ratio $O(k^2\cdot\ell^2)$.


\bibliographystyle{ACM-Reference-Format}
\bibliography{references}  

\begin{appendix}

\section{Ommited proofs}

\section{Todos}

\todo{Check naming consistency inside sections, and across them. Possibly contain some definitions in preliminaries.}

\todo{ACM format for review? Is there any specific format like this? With line numbers etc.}

\todo{Include e-mails? Or shall we skip it for now.}

\todo{Check ACM package whitelist}

\todo{Replace package algorithmpseudocode, it is not whitelisted https://www.acm.org/publications/taps/whitelist-of-latex-packages}

\todo{Make an ACM account for the submission}

\todo{ACM keywords are mandatory}

\todo{CCS concepts are mandatory}

\todo{Spellcheck, grammar (grammarly?)}

\todo{Make sure our manuscript is autor anonymous}

\todo{Font for OPT, ALG, DET, dist etc}

\todo{Line endings (add tildes where necessary)}
\end{appendix}

\end{document}
