 
\documentclass[manuscript,screen=true, review, anonymous]{acmart}

\usepackage[utf8]{inputenc}
\usepackage{xspace}
\usepackage{balance}
\usepackage{amsmath,amsfonts,mathtools,amsthm}
\usepackage{algorithmic}
\usepackage{algorithm}

\usepackage{balance}
\usepackage[english]{babel}
\usepackage{blindtext}
\usepackage{amsthm,amsmath,array,colortbl,graphicx,multirow}
\usepackage{comment}
\usepackage{balance}
\usepackage{tikz}
\usepackage{amsmath}
\usetikzlibrary{patterns} %
\usepackage{algorithm}
\usepackage[font={footnotesize}]{subcaption}
\usepackage[font={footnotesize}]{caption}
\usepackage{breakcites}
\usepackage{booktabs}
\usepackage{diagbox}
\usepackage{xcolor}
\usepackage{colortbl}
\usepackage{cleveref}
\usepackage{enumitem}

\mathchardef\mhyphen="2D

\title{Improved Bounds for Online Balanced Repartitioning}


\author{Maciej Pacut}
\email{maciej.pacut@univie.ac.at}
\orcid{0000-0002-6379-1490}
\affiliation{%
  \institution{Faculty of Computer Science, University of Vienna}
  \country{Austria}
}


\author{Mahmoud Parham} 
\email{mahmoud.parham@univie.ac.at}
\orcid{0000-0002-6211-077X}
\affiliation{%
  \institution{Faculty of Computer Science, University of Vienna}
  \country{Austria}
}

\author{Stefan Schmid} 
\email{stefan_schmid@univie.ac.at}
\affiliation{%
  \institution{Faculty of Computer Science, University of Vienna}
  \country{Austria}
}

\copyrightyear{2020} 
\acmYear{2020} 
\setcopyright{acmlicensed}
\acmConference{PODC '20}{July 29-August 2, 2020}{Toronto, Canada}

\keywords{online algorithms, competitive analysis, graph partitioning, clustering}
\acmISBN{}\acmPrice{}
\acmDOI{}

\ccsdesc[500]{Networks~Network algorithms}
\ccsdesc[300]{Computer systems organization~Cloud computing}
\ccsdesc[300]{Computer systems organization~Distributed architectures}

%%%%%%%%%%%%%%%%%%%%%%%%%%%%%%%%%%%%%%%%%%%%%%%%&&
%%%%%%%%%%%%%%%%%%%%%%%%%%%%%%%%%%%%%%%%%%%%%%%%&&
%  our macros start
%%%%%%%%%%%%%%%%%%%%%%%%%%%%%%%%%%%%%%%%%%%%%%%%&&
%%%%%%%%%%%%%%%%%%%%%%%%%%%%%%%%%%%%%%%%%%%%%%%%&&

\newcommand{\OPT}{\textsf{OPT}\xspace}
\newcommand{\ALG}{\textsf{ALG}\xspace}
\newcommand{\PPL}{\textsf{PPL}\xspace}
\newcommand{\OBRP}{BRP}
\newcommand{\PPOBRP}{PP-BRP}
\newcommand{\dist}{\textsf{dist}}
\newcommand{\TAlg}{{\ensuremath{\textsf{ALG}_{3}}}\xspace}

\newcommand{\comm}{\textsc{comm}}
\newcommand{\OFF}{\textsc{Off}\xspace}




\newtheorem{claim}{Claim}
\newtheorem{fact}{Fact}
\newtheorem{rem}{Remark}
\newtheorem{observation}{Observation}
\newtheorem{property}{Property}


\DeclarePairedDelimiter\pair{(}{)}
\DeclarePairedDelimiter\set{\{}{\}}

\DeclarePairedDelimiter{\ceil}{\lceil}{\rceil}
\DeclarePairedDelimiter{\floor}{\lfloor}{\rfloor}

\newcommand\mahmoud[1]{\color{green}\textbf{\\ Mahmoud: #1}\\\color{black}}
\newcommand\stefan[1]{\color{blue}\textbf{\\ Stefan: #1}\color{black}}
\newcommand\maciek[1]{\color{brown}\textbf{\\ Maciek: #1}\color{black}}


\newcommand{\todo}[1]{\noindent\color{brown}{todo: #1}\color{black}}

\begin{CCSXML}
	<ccs2012>
	<concept>
	<concept_id>10003033.10003068</concept_id>
	<concept_desc>Networks~Network algorithms</concept_desc>
	<concept_significance>500</concept_significance>
	</concept>
	<concept>
	<concept_id>10010520.10010521.10010537.10003100</concept_id>
	<concept_desc>Computer systems organization~Cloud computing</concept_desc>
	<concept_significance>300</concept_significance>
	</concept>
	<concept>
	<concept_id>10010520.10010521.10010537</concept_id>
	<concept_desc>Computer systems organization~Distributed architectures</concept_desc>
	<concept_significance>300</concept_significance>
	</concept>
	</ccs2012>
\end{CCSXML}

% for my small screen
%\setlength{\textwidth}{8cm}

\begin{document}

\begin{abstract}
Distributed   applications,  including  batch  processing, streaming, scale-out databases,
or machine learning, generate a significant amount of network traffic. By collocating frequently communicating nodes (e.g., virtual machines) on the same clusters (e.g., server or rack), the network load can be reduced and application performance improved. 
However, as the communication pattern is a priori unknown and may change over time, it needs to be learned efficiently, in an online manner.
%
This paper revisits the online 
balanced repartitioning problem 
(introduced by Avin et al.~at DISC 2016)
which asks for an algorithm that strikes
an optimal tradeoff between the benefits
of collocation (i.e., lower network load) 
and its costs (i.e., migrations). 
%
Our first contribution is a significantly improved
lower bound of $\Omega(k\cdot \ell)$ on the
competitive ratio, where $\ell$ is the number
of clusters and $k$ is the cluster size,
even for a scenario in which the communication
pattern is static and can be perfectly partitioned;
we also provide a tight upper bound 
of $O(k\cdot \ell)$ for this scenario.
In addition, we present a tight upper bound
of $\Theta(\ell)$ for $k=3$,
for the general model in which the
communication pattern can change arbitrarily
over time. 
\end{abstract}
    
\maketitle
    
\renewcommand{\shortauthors}{M.~Pacut, M.~Parham, S.~Schmid}

\section{Introduction}

\noindent \textbf{Motivation and model.}
The \emph{balanced repartitioning} problem (\OBRP{})
is a fundamental learning problem
that finds applications in the context of
distributed systems optimization~\cite{repartition-disc}. We are given a set $V$ of $n$ nodes 
(e.g., virtual machines or processes),
initially arbitrarily partitioned into $\ell$~clusters
(e.g., servers or entire racks),
each of size~$k$.
The nodes interact using
a sequence of pairwise communication requests
$\sigma = (u_1,v_1),$ $(u_2,v_2),$ $(u_3,v_3), \ldots$,
where a pair $(u_t,v_t)$ indicates that nodes $u_t$ and $v_t$ exchange a~certain amount of data.
Nodes in $C \subset V$ are \emph{collocated}
if they reside in the same cluster.

An algorithm serves a communication request between two nodes
either \emph{locally} at cost~0
if they are collocated,
or \emph{remotely} at cost~1
if they are located in different clusters.
We refer to these two types of requests as \emph{internal}
and \emph{external} requests, respectively.
Before serving a request,
an online algorithm may perform a \emph{repartition},
%(i.e., \emph{reconfigure}).
i.e.,
it may move (``migrate'') some nodes into clusters different from their current clusters, while respecting the capacity of every cluster. 
Afterwards, 
the algorithm serves the  request.
The cost of migrating a node from one cluster to another
is~$\alpha \in \mathbb{Z}^+$.
For any algorithm $\ALG$,
its cost,
denoted by $\ALG(\sigma)$,
is the total cost of communications and
the cost of migrations performed by $\ALG$ while serving the sequence $\sigma$.

We consider two flavors of the problem
in this paper. In the first one, which
was also studied by Henzinger et al.~\cite{sigmetrics19_partitioning}
(SIGMETRICS 2019) and which
we here mainly consider for the sake of a stronger
lower bound, we assume that $\sigma$
simply reveals the edges of a static graph
that can be perfectly partitioned
(i.e., in principle, no external requests
are required). 
In the second one, we consider a general
model where $\sigma$ can be arbitrary;
this was the original problem introduced
by Avin et al.~\cite{repartition-disc} at DISC 2016.
For simplicity, we will refer to the former
model as the \emph{learning} model (as
one has to learn the static communication graph) 
and to the latter as the \emph{online} model.


\noindent \textbf{Contributions.}
We provide an improved lower bound 
of $\Omega(k\cdot\ell)$ on the competitive ratio of any online deterministic online algorithm 
even in the learning model;
the best known lower bound so far was $\Omega(k)$,
even for a more general model~\cite{repartition-disc}.
We also present an asymptotically optimal, 
$O(k\cdot \ell)$-competitive algorithm
for the restricted model.
For the online model, we present  
an asymptotically optimal,
$\Theta(\ell)$ algorithm for $k=3$;
the best known upper bound 
so far was $O(\ell^2)$~\cite{repartition-disc}.
%
Our algorithms can be distributed
similarly to the approach in~\cite{sigmetrics19_partitioning}.
The overview of known results is in Table~\ref{tab:overview}.

\maciek{TODO: fill the table: results and references to sections.}

\begin{table*}
  \centering
  \renewcommand{\arraystretch}{1.5}
    \begin{tabular}{>{\centering\arraybackslash}p{4.5cm}|>{\centering\arraybackslash}p{4.5cm}>{\centering\arraybackslash}p{4.5cm}}
    \rowcolor{gray!50}
    \textbf{Variant} & \textbf{ Lower bound} &\textbf{Upper bound}\\ \hline 
      \textbf{$k=2$}& 3\hspace{0.3cm}\cite{repartition-disc} & TBD\hspace{0.3cm}(\S 9.3) \\ 
      \rowcolor{gray!25}
      \textbf{$k=3$}&  $27\cdot \ell $\hspace{0.3cm}(\S 9.3)& $O(\ell)$\hspace{0.3cm}(\S 9.3)\\
      \textbf{$\ell=2$}&  $k+1 = \Omega(k)$\hspace{0.3cm}(\S 9.3)&\hspace{0.3cm}$O(k^2)$\hspace{0.3cm}\cite{repartition-disc}\\
      \rowcolor{gray!25}
      $\ell \geq 3$& $\Omega(k\cdot \ell)$\hspace{0.3cm}(\S 9.3)&$O(k^2 \cdot \ell^2)$\hspace{0.3cm}\cite{repartition-disc} \\
      Perfect partition & $\Omega(k\cdot \ell)$\hspace{0.3cm}(\S 9.3)&$O(k \cdot \ell)$\hspace{0.3cm} (\S 9) \\
    \end{tabular}
                \caption{Overview of the results. Most specific cases are tight. The general case still has a significant gap between the lower bound and the upper bound. If either $k$ or $\ell$ is missing from a variant description, it is assumed to be arbitrary.
                  }
  \label{tab:overview}
  \vspace{-7mm}
\end{table*}

\noindent \textbf{Related work.}
The existing results in~\cite{repartition-disc}
and~\cite{sigmetrics19_partitioning}
primarily focus on a model where 
the online algorithm can use augmentation,
i.e., has slightly larger clusters than the offline 
algorithm. In contrast, in our paper, we focus
on a scenario where the nodes need to be
perfectly balanced among the clusters.
The problem has also been studied in a weaker
model where the adversary can only sample
requests from a fixed distribution~\cite{stochastic-ring}.
For clusters of size $2$ (i.e., $k=2$), 
it is known that a constant competitive algorithm exists~\cite{repartition-disc}.

%More generally, the model is related to online
%caching~\cite{SleTar85,FKLMSY91,McGSle91,AcChNo00},
%see~\cite{repartition-disc} for a discussion.
%The static offline version of~the~problem, called the
%\emph{$\ell$-balanced graph partitioning problem} is 
%NP-complete, and cannot even be approximated within %any finite factor unless P
%= NP~\cite{AndRae06}. 

\noindent \textbf{Preliminaries.}

\maciek{Here, we need a good and broad description of what a component is.}

In this paper, we use the technique of maintaining \emph{(connected) components} of nodes, similar to previous approaches~\cite{repartition-disc}. A~\emph{component} is a subset of frequently communicating nodes \maciek{This is not a definition, just some intuitions. More formal needed.}.
We use this concept in both algorithms in this paper, but also in the lower bound, to incur high cost for any online algorithm that splits them.
An~algorithm is \emph{component respecting}
if it always keeps nodes that  belong to the same component collocated.
That is,
if the algorithm needs to move a node,
it moves the whole component containing the node.
%We say that the request $(u,v)$ is \emph{external}, if the algorithm is in the configuration, where $u$ and $v$ are located at different clusters.
%This is achieved by opting a sequence of  partitions (a.k.a.~configurations)
%such that frequently communicating nodes (from different clusters)
%are relocated to the same cluster before they inflict too much communication cost.
%A \emph{reconfiguration} (i.e.,\emph{ re-partitioning})
%is performed by migrating nodes between clusters,
%at the total cost of individual node migrations.
%The objective is to jointly minimize communication and migration costs while serving the sequence.

\maciek{Let's follow the SIDMA notation here. For example, a "single" node (the one that belongs to singleton component) is called "trivial" there.}


\section{The Learning Problem} %$\Omega(k\cdot \ell)$ for Competitive Ratio of Any Deterministic Algorithm}

\maciek{TODO: Intro to the learning problem is missing. What is the problem definition. How is it different from the general problem. Emphasize the fact that OPT is constrained not to migrate.}

\subsection{Lower bound.}

\label{sec:lowerbound}

\maciek{The text before the proof should have two parts: a description of adversary components and then general intro to the proof.}

In this section, we provide a lower bound $\Omega(k\cdot \ell)$ for the competitive ratio of any deterministic online algorithm.
It is sufficient to have clusters of capacity $k\geq 3$ for this lower bound.
For $k=2$, a constant competitive algorithm exists.
At the end of this section, we also
discuss an upper bound.

\maciek{TODO: rename everywhere: "adversary's components", i.e., all components in lower bounds should be called "ground sets"}


\maciek{Example figure of the construction and growing components.}


The adversary constructs components of nodes based on the partitioning choices of a given online algorithm.
Adversary issues requests between pairs of nodes that belong to the same component.
If an online algorithm splits a component at any moment,
the adversary issues requests to the split pair until the algorithm collocates them.
We utilize the fact that the online algorithm does not know the components of the adversary, and it may split some components while evicting nodes.
On the other hand, $\OPT$ initially collocates all nodes of each component and never splits them again.


\maciek{Here, we would like to introduce insights about the proof, emphasize the conditions for the bound to work (such as $k\geq 3$), and maybe compare to previously known lower bounds. Also, introduce the notation and definitions that are needed.}

\maciek{Here: we abstract from the value of $\alpha$, but works for any alpha.}

\begin{theorem}
  The competitive ratio of any deterministic online algorithm for \OBRP{} is in $\Omega(k\cdot \ell)$ for any $k\geq 3$.
  \maciek{TODO: the exact bound, not asymptotic}

  \maciek{The bound $(k \cdot \ell - k - \ell) / 2$ can be easily improved to $(k \cdot \ell - k - 1) / 2$ by playing a simple caching game in the first cluster, instead of just giving away the $k-1$ "blocker" (the caching game would build it gradually)}
\end{theorem}

\begin{proof}
	We construct an instance of the problem with $\ell$ clusters 
	$\set{ S_1, S_2,\dots , S_{\ell}}, |S_i|  = k$.
	Let $I(C)$ denote the cluster where nodes of component $C$ are located initially.
	%	\maciek{So far components were not introduced; the best place to introduce these would be around the definition of \PPOBRP}.
  Fix any online algorithm \ALG{}.
  
  \maciek{This paragraph should be before the proof, merged with section about adversary components.}
  We join some nodes into components, and produce the input sequence by observing actions of \ALG{} with respect to nodes of these components.
	If \ALG{} at any point splits a component
	(i.e., spreads its nodes over two or more clusters),
	then we continue to issue requests between non-collocated nodes of the component until \ALG{} collocates the nodes of the component.
	If \ALG{} never collocates them, then it is not competitive.

  Initially each node belongs to its own component of size $1$.
  We say that a node is single if it belongs to a singleton component.
  We begin by forming a component of single $k-1$ nodes on the cluster $S_1$.
  
	Let $x_0$  be the only single node left on $S_1$ and  $y_0 \in S_2$ be any single node on $S_2$.
	Next,
	we merge components of $x_0$~and~$y_0$.
  Since these reside on different clusters of \ALG{}, it produces an external request.
  \ALG{} must collocate them in a~cluster other than $S_1$.
	To this end,
	\ALG{} moves to a new configuration
	that replaces $x_0$ and $y_0$ with two other single nodes $x_1$ and $y_1$ respectively.
	
	Next,
	we merge the component of $x_1$ and the largest component $C$ s.t.~$I(C) = I(x_1)$.
	\ALG{} must collocate $x_1$ and $C$ in a~cluster other than $S_1$ and
	consequently replace $x_1$ with some other (single) node $x_2$.
	We repeat merging the component of single node $x_i$ on $S_1$ and the largest component $C'$ s.t.~$I(C')=I(x_i)$,
  until there are only two single nodes left that  originate from the same cluster.
	Formally, we denote these two single nodes by $x^*, y^*$, and we have $I(x^*) = I(y^*)$, and~for any other pair of single nodes
	$x'$ and $y'$,
	we have $I(x') \neq I(y')$.
	At this point there are at most $\ell+1$ single nodes left,
	otherwise, there would be more pairs of single nodes that were initially in the same cluster.
	
	Given this sequence of requests,
	the optimal strategy is to migrate $\set{x_0,y_0}$ to the cluster $I(x^*)$ by
	swapping $\set{x_0,y_0}$ with $\set{x^*,y^*}$.
	Hence,
	OPT pays for $2$ node swaps and
	\ALG{} incurs at least one swap for each node in the sequence $X := x_0, x_1,\dots$.
	We exclude at most $(k-1) + ( \ell+1)$ nodes out of $k \cdot \ell$ nodes,
  therefore $|X| \geq k \cdot \ell - k - \ell$ \maciek{elaborate on exclusion?}.
  Finally, $\ALG/\OPT \geq (k \cdot \ell - k - \ell) / 2\in \Omega(k\cdot\ell)$.
\end{proof}

\maciek{Remark on resource augmentation. The lower bound for k=3 works even with significant augmentation (linear).}

\subsection{Upper bound.}
Now we give an algorithm for a restricted variant of  \OBRP{}, named \PPOBRP{},
%\maciek{This is not sufficient! We also disallow serving requests remotely.}
where an optimal offline algorithm ($\OPT$) moves to a perfect partition
at the beginning and stays there perpetually.
\maciek{Already mention that the LB is for that scenario}
%We refer this variant as \PPOBRP{}.
The task of an online algorithm for \PPOBRP{} is to recover (or learn) the perfect partition while not paying too much relative to \OPT.
%
%\maciek{This is not a valid description of the model. We need the assumption that collocation is needed in our model. Otherwise OPT would just serve some requests remotely, and our analysis would not hold.}
The first request between any two non-collocated nodes reveals an edge of the communication graph.
Our online algorithm collocates them immediately and never separates them since by assumption \OPT does the same.
Revealed edges, up to any point in the sequence,
induce some connected subgraphs which grow as components \maciek{rewrite sentence, expand}.
%An algorithm that always  keeps nodes belonging to the same component collocated is \emph{component respecting}.

We assume \OPT begins with the initial configuration
$P_I = I_1, \dots, I_{\ell}$ and moves to the final partitioning
$P_F = F_1, \dots, F_{\ell}$.
 The \emph{distance} of a configuration $P = C_1, \dots, C_{\ell}$ from the initial configuration \maciek{, denoted $\dist(P)$} is the number of nodes in $P$ that do not reside in their initial cluster.
 \maciek{It is OK to use single-argument $\dist$ (two-arg is a must in a definition, and it is already there).}
 \maciek{We should denote distance $\Delta(P)$, and the current $\Delta$ should be R (like radius).}
    That is,
    $\dist(P, P_I) := \sum_{j=1}^{\ell} | C_j \setminus I_j |$. 
In other words,
at least $\dist(P, P_I)/2$ node swaps are required in order to reach the configuration $P$ from $P_I$, and thus
$\OPT \geq \Delta:= \dist(P_F, P_I) $.
 With each repartitioning,
  \PPL moves to a configuration that minimizes the distance to the initial configuration $P_I$.
As a result,
\PPL never ends up in a configuration that is more than $\Delta$ (single-node) migrations away from $P_I$.
This invariant ensures that \PPL does not pay too much while recovering $P_F$.
%      \maciek{``re-partition'' procedure name should indicate the fact that this is a specific repartition that is close to initial partition}
      We note that the $\mathit{repartition}$ at Line \ref{line:rebalance} replaces the current configuration $P$ with a~perfect partition closest to $P_I$.
Hence it never moves to a configuration beyond distance $\Delta$.
The scheme of the algorithm can be found in the Algorithm \ref{alg:PPL}.

\section{PPL Algorithm} \label{alg:PPL}
\begin{algorithm}
	\renewcommand{\algorithmicrequire}{\textbf{Input:}}
	\renewcommand{\algorithmicensure}{\textbf{Output:}}
	\begin{algorithmic}
		%        \Require 
		%        $k, \ell$,
		%        initial configuration $P_I$,
		%        sequence of  requests $\sigma_1, \dots, \sigma_N$ 
		%        \Ensure A final configuration $P_F$ 
		\STATE {For each node $v$ create a singleton component $C_v$ and add it to $\mathcal{C}$}
		\STATE{$P_0 := P_I$}
		\label{line:initcomponents}
		\FOR {each  request $\sigma_t=\{u,v\}, 1 \leq t \leq N$}
		\STATE Let $C_1 \ni u$ and $C_2 \ni v$ be the container components
		\IF{$C_1 \neq C_2$}
		\STATE {Unite the two components into a single component $C'$ and
			$\mathcal{C} = (\mathcal{C}\setminus\set{C_1, C_2}) \cup ~\set{C'}$} \label{line:mergecomponents}
		\IF{$\mathit{cluster}(C_1, P_{t-1}) \neq \mathit{cluster}(C_2, P_{t-1})$
			\COMMENT{i.e.~if not in the same cluster}    
		}       
		\STATE {$P_{t} = \mathit{repartition}(P_{t-1}, P_I, \mathcal{C})$} 
		\COMMENT{move to $P$ closest to $P_I$}
		\label{line:rebalance} 
		\ENDIF
		\ENDIF
		\ENDFOR
	\end{algorithmic}
	\caption{Perfect Partition Learner (\PPL)}
	\label{alg:ppl}
\end{algorithm}

\begin{property} \label{prop:dist<OPT}
    Let $P$ be any configuration chosen by \PPL at Line $\ref{line:rebalance}$.
    Then, $\dist(P,P_I) \leq \Delta$.
\end{property}

\begin{lemma}	\label{lemma:rebalancecost}
    The cost of repartitioning at Line \ref{line:rebalance} is at most $2\cdot\OPT$.
\end{lemma}
\begin{proof}
    Consider the repartitioning that transforms $P_{t-1}$ to $P_t$ upon the request $\sigma_t$.
    Let $M \subset V$ denote the set of nodes that migrate during this process.
	Let $M^-$ and $M^+$ denote the subset of nodes that (respectively)
    enter or leave their original cluster during the repartitioning.    
    Then,
    $M = M^+ \cup M^-$.
    Since $|M^-|$ nodes are not in their original cluster before the repartitioning (i.e., in $P_{t-1}$),
    the distance before the repartitioning is $\dist(P_{t-1},P_I) \geq | M^-|$.
    Analogously,
     the distance afterwards is $\dist(P_{t},P_I) \geq | M^+|$.
    Thus,
    $|M| \leq \dist(P_{t-1},P_I) + \dist(P_{t},P_I)$.
    By Property \ref{prop:dist<OPT},
    $\dist(P_{t-1},P_I) , \dist(P_{t},P_I) \leq \Delta \leq \OPT$
    and thereby we have	
    $|M| \leq 2\cdot\OPT$.
    \maciek{the comma between $\dist$ is confusing. BOTH are bounded?}
\end{proof}

\begin{theorem}	\label{thm:upperbound}
    \PPL reaches the final configuration $P_F$ and it is $(2\cdot k\cdot\ell)$-competitive.
    \maciek{Do not hide that we have $2(k-1)\ell$ in the theorem. This is a better bound.}
\end{theorem}
\begin{proof}
      On each inter-cluster request,
     the algorithm enumerates all $\ell$-way partitions of components
     that are in the same (closest) distance of $P_I$.
     That is, 
     once it reaches a configuration $P$ at distance $\Delta = \dist(P, P_I)$,
     it does not move to a configuration
     $P', \dist(P', P_I) > \Delta$,
     before it enumerates all configurations at distance $\Delta$.
     Therefore,
     \PPL eventually reaches $\Delta=\OPT$ and the configuration $P_F$.
%    including the request that completes revealing of all components that are collocated in $P_F$.
    There are at most $(k-1)\cdot\ell < k\cdot\ell $ calls   to $\mathit{repartition}$
     (i.e., the number of internal edges in $P_F$).
     \maciek{Clarify why the number of internal edges bound the number of repartitions.}
    By Lemma \ref{lemma:rebalancecost},
    each repartition costs at most $2\cdot\OPT$.
    The total cost is therefore at most $2\cdot\OPT\cdot k\cdot\ell$, which implies the competitive ratio.
%    \mahmoud{This is the cost of moving and the cost of remote comm. is not counted.
%    	So it is 4-competitive (?)}
 \end{proof}

% In Section~\ref{sec:lowerbound} we constructed a $\Omega(k \cdot \ell)$ for \OBRP{}.
% Note that the lower bound holds also in the perfect partition model, as the constructed input sequence allows \OPT to move to a perfect partition.
% The corollary is that PPL is optimal.
 
\section{The Online Problem}
\label{sec:k3}

\maciek{REALLY bad name. Both the learning problem and the general one is online.}

Let us now discuss the general online
model where the request sequence
can be arbitrary. We show that the classic \emph{rent-or-buy} approach~\cite{karlin-ski-rental} allows obtaining an optimal algorithm for $k=3$.
Upon receiving $\alpha$ requests between a pair of nodes, we collocate them in one cluster until the end of a phase (that we define precisely later).

%We define the algorithm \TAlg in the following way.
Now we describe the algorithm \TAlg.
It partitions nodes into components, and
initially, each node belongs to its own singleton component.
For each pair of nodes, \TAlg maintains a counter, initiated to $0$. 
Upon receiving a request to a pair that is not collocated in one cluster, it increases their counter by $1$.
If the counter for a pair $(u,v)$ reaches $\alpha$, \TAlg merges the components of $u$ and $v$, and moves to the closest component respecting partitioning.
If no such partitioning exists, \TAlg resets all components to singletons, resets all counters to $0$, and ends the phase.



\maciek{Figure: cluster signatures}

\maciek{Figure: an unsaturated edge that is inside a component}


\maciek{Rewrite: cluster types should be cluster signatures}
In our analysis, we distinguish among three types of clusters: $C_1, C_2, C_3$. In a cluster of type $C_i$, the size of the largest component contained in this cluster is $i$.
Before bounding the competitive ratio of \TAlg, we introduce the lemma that estimates the cost of a single repartition of \TAlg.

\begin{lemma}
  \label{lem:1req}
  During a single repartition, \TAlg exchanges at most $2$ pairs of nodes.
\end{lemma}

\begin{proof}
    Observe that when the repartition is triggered by \TAlg, the resulting partitioning is component respecting.
  Otherwise, if it does not exist, \TAlg simply ends the phase and performs no repartition.

  %We distinguish between three types of clusters: $C_1, C_2, C_3$,
  %which we define as follows.
  %A cluster of type $C_1$ the cluster contains $3$ singleton components (this is also the initial configuration of any cluster).
  %A cluster of type $C_2$  contains one component of size $2$ and one component of size $1$.
  %Finally, a cluster of type $C_3$  contains one component of size $3$.
  
    Consider a request between $u$ and $v$ that triggered the repartition and let $U$ and $V$ be their respective clusters.
    Note that $U\neq V$,
	 as otherwise, the request would not trigger a~repartitioning.
	 We consider cases based on the types of clusters $U$ and $V$.
    Note that this is impossible that either $U$ or $V$ is of type $C_3$, as otherwise, we merge components of size $3$, and no component respecting partitioning exists, a contradiction.
    If either $U$ or $V$ is of type $C_1$, then this cluster can fit the merged component, and the reconfiguration is local within $U$ and $V$.
    For two clusters, any configuration can be reached within two swaps, due to the fact that clusters are indistinguishable.
  
    Finally, we consider the case where both $U$ and $V$ are of type $C_2$. Note that $(u,v)$ cannot both belong to a component of size $2$, as this would mean that \TAlg has the component of size $4$, a contradiction with the case assumption that the component respecting partitioning exists. 
    Otherwise, if either $u$ or $v$ belongs to a component of size $2$, then it suffices to exchange components of size $1$ between $U$ and $V$.
    Finally, if $u$ and $v$ belong to components of size $1$, then we must place them in a cluster different from $U$ and $V$.
    Note that in such case, a $C_1$-type cluster $W$ exists, as otherwise no component respecting partitioning exists. In this case \TAlg performs one swap, i.e., it exchanges the nodes $u$ and $v$ with any two nodes of $W$.
\end{proof}

\begin{theorem}
  \TAlg is $O(\ell)$-competitive.
\end{theorem}
\begin{proof}
  Fix a completed phase, and consider the state of \TAlg's counters at the end of it.
  We consider the incomplete phase later in this proof.

  As \TAlg is component respecting, it never increases any counter above $\alpha$.
  We say that the pair $(u, v)$ is \emph{saturated} if the counter has value $\alpha$, and \emph{unsaturated} otherwise \maciek{Rename: saturated/unsaturated edge instead of a pair}.
  Let $\sigma$ be the input sequence that arrived during the phase.
  Let $\sigma_{cost}$ be the requests that at the moment of arrival were external requests for \TAlg (these are the only requests that incurred a cost for \TAlg).
  In our analysis, we partition $\sigma_{cost}$ into subsequences $\sigma_I$ and $\sigma_E$.
  The sequence $\sigma_I$ (inter-component requests) are the requests from $\sigma_{cost}$ issued to pairs that belong to the same component of \TAlg at the end of the phase.
  The sequence $\sigma_E$ (extra-component requests) denotes the requests from $\sigma_{cost}$ that do not appear in $\sigma_I$.


  %Let $A^I$ be the cost of (extra-cluster) communication incurred in this phase by \TAlg between pairs that belonged to the single component at the end of the phase.
  %Let $A^E$ be the cost of (extra-cluster) communication incurred in this phase between the nodes that belong to different components at the end of this phase.
  Let $\TAlg(M)$ be the cost of migrations performed by \TAlg in this phase.
  \TAlg performs at most $2 \ell$ component merge operations, as
  exceeding this number means that a component of size $4$ exists, and the phase should have ended already.
  Combining this with Lemma~\ref{lem:1req} gives us $\TAlg(M) \leq 8\alpha\cdot\ell$ (recall that an exchange costs $2\alpha$).
  %Together with Lemma~\ref{lem:1req}, this allows us to bound the cost of migrations, $\TAlg(M) \leq 6\cdot\alpha\cdot\ell$.
  
  Now we bound $\TAlg(\sigma_I)$.
  A~cluster of type $C_3$ contributes at most $3 \alpha - 1$ to $\TAlg(\sigma_I)$, as $2$ of pairs of nodes from the component are saturated and contribute $\alpha$ each, and the third, unsaturated pair contributes at most $\alpha-1$.
  Other cluster types contribute less: $C_1$ contributes $0$ and $C_2$ contributes $\alpha$.
  Summing this over all $\ell$ clusters gives us $\TAlg(\sigma_I) \leq (3 \alpha-1)\cdot \ell \leq 3\alpha\cdot\ell$.

  %We bound $A^E$ by $k^2 \cdot (\alpha - 1)$, as no more than $k^2$ pairs are unsaturated, and each of them contributes at most $\alpha -1$.
  %\maciek{not needed most likely}

  Moreover, \TAlg paid for all requests from $\sigma_E$, and thus $\TAlg(\sigma_E) = |\sigma_E|$.
  In total, the cost of \TAlg is at most $\TAlg(\sigma_I) + \TAlg(\sigma_E) + \TAlg(M) \leq 11\alpha\cdot \ell + |\sigma_E|$ during this phase.

  \medskip

  Now we lower-bound the cost of $\OPT$.
  By $\OPT(\sigma_I)$ and $\OPT(\sigma_E)$ we denote the cost of $\OPT$ on these input sequences (defined with respect to components of \TAlg in this phase).
  By $\OPT(M)$ we denote the cost of migrations performed by $\OPT$ in this phase.
  
  We split the cost of $\OPT$ into parts coming from serving $\sigma_I$ and $\sigma_E$.
  While serving these requests, $\OPT$ may perform migrations, and we account for them in both parts: we separately bound $\OPT$ by $\OPT(\sigma_I) + \OPT(M)$ and $\OPT(\sigma_E) + \OPT(M)$.
  Combining those bounds gives us $\OPT \geq \max\{\OPT(\sigma_I) + \OPT(M), \OPT(\sigma_E) + \OPT(M)\} \geq (\OPT(\sigma_I) + \OPT(M)) / 2 + (\OPT(\sigma_E) + \OPT(M)) / 2$.

  %Let $O^M$ be the cost of migrations performed by $\OPT$ during the phase.
  %Let $O^I$ be the cost of serving requests between nodes that were put in one component by \TAlg during this phase.
  %Let $O^E$ be the cost serving requests between nodes that \TAlg did not put in the same component during that phase.

  %First, we estimate the cost related to $\sigma_I$.
  We have $\OPT(M) + \OPT(\sigma_I) \geq \alpha$, as the phase ended when the components of \TAlg{} could not be partitioned without splitting them.
  Hence, for every possible configuration of $\OPT$, there exists a non-collocated pair of nodes with at least $\alpha$ requests between them, and
  $\OPT$ either served them remotely or performed a~migration.

  \medskip
  Before we bound the competitive ratio, we relate the costs of $\TAlg$ and $\OPT$ with respect to requests $\sigma_E$.
  In a~fixed configuration of $\OPT$, it may mitigate paying for requests between at most $3\ell$ pairs of nodes by collocating them in its clusters.
  Recall that $\sigma_E$ consists of requests to unsaturated pairs, and it accounts only for requests that increased the counter (i.e., external requests), thus $\OPT$ may mitigate at most $3\ell\cdot(\alpha - 1)$ requests from $\sigma_E$.
  Faced with $W := |\sigma_E| - 3\ell\cdot(\alpha-1)$ requests $\sigma_E$ that it could not mitigate, $\OPT$ served some of them remotely and possibly performed some migrations to decrease its cost.


  Now we estimate the cost of $\OPT(\sigma_E)$ while accounting savings from migrations.
  By performing a swap of nodes $(u,v)$, $\OPT$ collocates $u$ with two nodes $u', u''$, and $v$ with two nodes $v'$, $v''$.
  This may allow serving requests between $(u,u')$, $(u,u'')$, $(v,v')$ and $(v,v'')$ for free afterwards.
  As $\sigma_E$ consists of requests to unsaturated pairs, and it accounts only for external requests, there are at most $\alpha-1$ requests between each of these pairs.
  By performing a single swap that costs $2\alpha$, $\OPT$ may avoid paying the remote serving costs for at most $4 (\alpha - 1)$ requests from $\sigma_E$.
  Thus, for serving $\sigma_E$, $\OPT$ pays at least $\OPT(\sigma_E) + \OPT(M) \geq W \cdot \frac{2\alpha}{4 (\alpha-1)}\geq |\sigma_E| / 2 - 2 \alpha \cdot \ell$.
  From that, we obtain $|\sigma_E| \geq 2(\OPT(\sigma_E)+\OPT(M)) + 4\alpha \cdot \ell$.
  Let $E := \OPT(\sigma_E) + \OPT(M)$. Finally, we are ready to bound the competitive ratio
  \begin{equation*}
    \frac{\TAlg(\sigma)}{\OPT(\sigma)} \leq \frac{11\alpha \cdot \ell + |\sigma_E|}{\alpha/2 + E/2} \leq \frac{27\alpha\cdot\ell + 2\cdot E}{\alpha + E} \leq 27 \ell = O(\ell).
  \end{equation*}

  \medskip

  \maciek{Big techincal TODO. Formalize the sketched proof for the last phase.}
  It remains to consider the last, unfinished phase.
  Consider the case, where the unfinished phase is also the first one.
  Then, we cannot charge $\OPT$ due to the inability to partition the components.
  Instead, we use the fact that \TAlg and $\OPT$ started with the same initial configuration.
  We charge $\OPT$ $\alpha$ for the first external $\alpha$ requests or a migration,
  and we follow the analysis regarding the unsaturated requests.
  If the input finished before the first $\alpha$ external requests, then \TAlg is optimal.
  Now, consider the case, where there are at least two phases, then we split the cost $\alpha$ charged in the penultimate phase into last two phases, and follow the analysis regarding the unsaturated requests.
  This way, the competitive ratio increases at most twofold.
\end{proof}


\bibliographystyle{ACM-Reference-Format}
\bibliography{references}  

\pagebreak
\appendix


\section{Omitted results}

\begin{enumerate}
  		 \item The lower bound of $\Omega(\ell)$ for the scenario with at most $(1/3-1/k)$-augmentation
  
  		 \item Costly cascade example (it is easy to present $k^2$ and $k\cdot \sqrt{k}$ (with different approach), and then we can combine these to obtain $k^3$)
  
  		 \item $k$-competitive algorithm for the ring communication pattern
  
  \end{enumerate}

\section{Maciek's sketches}

\subsection{Cascade bounds and an improved algorithm for general k, l}

For now, assume $\alpha=1$ (not to deal with inter-component communication).
Consider an algorithm DET2, a modified version of DET from \cite{repartition-disc}.
Modification is the following: when repartition is necessary, move to \emph{the closest} perfect partition of components.

Now we bound the cost of a single repartition of DET2 independent of $\ell$.
First, we bound the cycle length in one of optimal solutions.

\begin{theorem}
  \label{th:cascade-cycles}
  Consider a single repartition of DET2 caused by node merge.
  There exists a sequence of node migrations that realizes this repartition, where each node-disjoint cycle has length at most $k$.
\end{theorem}
Describe the graph. Edges and labels.

Proof by contradiction.
We call a sequence of node migrations that realize the repartition \emph{a solution to repartition}.
Assume that each cost-optimal solution to repartition contains a cycle of length at least $k+1$.

Fix a solution to repartition that has the shortest length of the longest cycle, and has the least edge-disjoint cycles of such length.

\begin{proof}
\end{proof}


\begin{theorem}
  Consider a single repartition of DET2.
  The number of clusters it involves is bounded by $k^k$.
  \label{th:cascade}
\end{theorem}

\begin{proof}
  Each node has degree at most $k$. Combine with Theorem \ref{th:cascade-cycles}.

Important technical details:
\begin{itemize}
  \item There are no cycles attached to another cycles by just one vertex (by equivalent signature skipping). This is important, as otherwise we just have a bound on cycles, but the graph contains a chain of cycles.
  Another formulation (Stefan's): each edge is a part of a cycle that goes through the "origin" part of the graph.
\end{itemize}
\end{proof}


\begin{theorem}
  Algorithm DET2 is $O(\ell \cdot k\cdot k^k)$-competitive.
\end{theorem}

\begin{proof}
  Use Theorem~\ref{th:cascade}. Then, linearize the inner-component communication using the technique from SIDMA.
\end{proof}



\subsection{Perfect Partition reachable within one swap}

\maciek{Best to include it in the improved LB omega(lk) to help increase the lb by k-1}

  (The version for two clusters, but extensions possible.)
  Results: LB = UB = k+1.

  In this section we show that the competitive ratio of the BRP variant, where the perfect partition is reachable within one swap is exactly k+1.

  Previously the best upper bound was $O(k^2)$ of DET \cite{repartition-disc}, and the best lower bound was $k-1$, utilizing the reduction from paging from \cite{repartition-disc}.
  \maciek{It is possible to improve the reduction, i.e., to reduce from paging with cache $k$, but it is impossible to have reduction from paging with cache $k+1$.}

  \begin{theorem}
    If the perfect partition is reachable within one swap, the cost of any deterministic algorithm is at least k+1.
  \end{theorem}

  \begin{proof}
    \maciek{pre-sketch}
    A request to any pair. The algorithm puts it on the wrong cluster, evicting some node and paying $1$.
    We continue to issue the requests to the evicted node (it changes every time), alg pays $k-2$.
    We issue the final request, and alg must move the initial pair, paying $2$.
  \end{proof}

  \begin{theorem}
    There exists a (k+1)-competitive algorithm for the BRP variant, where the perfect partition is reachable within one swap.
  \end{theorem}
  
  \begin{proof}
    \maciek{Maybe PPL can be used instead?}

    \maciek{early sketch}
    The algorithm maintains components of nodes, and never splits them.
    Intially all nodes are in singleton components.
    The only choice is where to place the component, and possibly which components to evict.

  The algorithm places the first component, denoted $Z$, on an arbitrary cluster, and evicts an arbitrary node (it is a singleton).
  Consider a request $(a,b)$.
  If $a \in Z$ (also symmetry $b \in Z$), then we move the entire component $Z$ to the cluster $I(b)$, evicting a singleton node that is missplaced (i.e., is not on its initial cluster), and an arbitrary node (we later show that there exists a node that is misplaced).
  Otherwise, $I(a) = I(b)$, and we place the component there, evicting an arbitrary node (non-missplaced?).

  If a component collocated with $Z$ grows to $k-1$, we evict $Z$ and terminate.
  \end{proof}


\subsection{Lower bound $2k-1$}



\maciek{Dont include in WAOA}

We show the reduction from the paging problem with the universe $U$ of size $2k$ and the cache of size $2k-1$.
Recall that the lower bound for the competitive ratio for such paging problem is $2k-1$.

Similarly to the lower bound TODO ref $\Omega(kl)$, we construct components of nodes, and issue the requests if the algorithm splits the component.
In the proof in this section, the components produced are oblivious to the configuration of the algorithm, yet the requests issued depend on if the algorithm splits the components. In $\Omega(kl)$ LB, the components were constructed upon the actions of the algorithm, too.

\maciek{TODO: figure of three states of OPT. It should fit next to the other lower bound figure?}

\begin{theorem}
  %No deterministic algorithm for BRP with two clusters can achieve a competitive ratio smaller than $2k-1$.

  Fix any $k$. If there exists a $\gamma$-competitive deterministic algorithm $B$ for Dynamic Bisection Problem with cluster size $k$, then there exists a $\gamma$-competitive deterministic algorithm $P$ for the paging problem with cache size $2k-2$ and where the number of different pages is $2k-1$. 
\end{theorem}

\begin{proof}
  The pages are denoted by $p_1,p_2,\ldots,p_{2k-1}$. Without loss of generality, we assume that the initial cache is equal to $p_1,p_2,\ldots,p_{k-2}$.
  We fix any input sequence $\sigma^P = (\sigma^P_1, \sigma^P_2, \sigma^P_3, \ldots)$ for the caching problem, where $\sigma^P_t$ denotes the $t$-th accessed page.
  We show how to construct an online algorithm $P$ for the caching problem that proceeds in the following online manner.
  The algorithm internally runs the algorithm~$B$, starting on the initial assignment of nodes to clusters that will be defined below.
  For a requested page $\sigma^P_t$, it creates a subsequence of~communication requests for the BRP problem, runs $B$ on them, and serves $\sigma^P_t$ on the basis of $B$'s responses.

  We use $2k$ nodes for the BRP problem: $2k-1$ caching nodes denoted $p_1,p_2, \ldots, p_{2k-1}$, and an auxiliary node $a$.
  We partition nodes into sets $A = \{ a, p_1 \}$, $B = \{ p_2, p_3, \ldots, p_{k+1}\}$, and $C = \{ p_{k+2}, p_{k+3} \ldots, p_{2k-1}\}$.
  Their cardinalities are $|A| = 2$ and $|B| = |C| = k-1$.

  We say that the node configuration is \emph{well-aligned} if any of the following conditions hold:
  \begin{enumerate}
    \item One cluster contains $B$ and one node from $A$ and the other cluster contains $C$ and the other node from $A$.
    \item One cluster contains $B$ and one node from $C$ and the other cluster contains $A$ and $k-2$ nodes from set $C$.
    \item (symmetric to the previous condition) One cluster contains $C$ and one node from $B$ and the other cluster contains $A$ and $k-2$ nodes from set $B$.
  \end{enumerate}
  We show a bijection between cache contents and well-aligned configurations.
  We say that a caching node $p_i \in S$, where $S \in \{ A, B, C\}$ belongs to the cache if the entire set $S$ is collocated in a cluster.
  Otherwise, if $p_i$ is located on a different cluster than the other nodes from $S$, we say that it is outside of the cache.
  (Without loss of generality, we may assume that the cache of any caching algorithm is always full, i.e., consists of $k-1$ pages.)
  If the configuration $c$ of a~BRP algorithm is well-aligned, $\textsc{cache}(c)$ denotes the corresponding cache contents.
  The initial configuration for the BRP problem is the well-aligned configuration corresponding to the initial cache (pages $p_1,p_2,\ldots,p_{2k-2}$ in the cache), with sets $A$ and $B$ non-split and the node $p_{2k-1}$ from the set $C$ split.

\medskip

When a request $p$ to paging arrives, we create a subsequence $\comm(p)$ of communication requests for the BRP problem.
We identify the corresponding caching node $p$ and the set $S \in \{ A,B,C \}$ that contains it.
The construction of $\comm(p)$ depends on the actions of $B$.
To abstract from the actions of $B$ in the construction, we use the notion of the adversary components.
Forming an adversary component means that the adversary continuously issues requests to any two nodes of a component that are split.
The request sequence $\comm(p)$ consists of the requests the adversary issues for $B$ after forming an adversary component $S$.
When $B$ collocates all nodes of $S$, the adversary dismisses the adversary component $S$, i.e., no longer issues requests if $B$ splits the nodes of $S$.

Note that $B$ must eventually achieve such a node configuration: otherwise its cost would be arbitrarily large while a sequence of repeated $\comm(\sigma^P_t)$ subsequences can be served at~a~constant cost.
Therefore, the competitive ratio of~$B$ would then be unbounded.
We denote the resulting sequence of $\comm(\sigma^P_t)$ subsequences by $\comm_t(\sigma^P_t)$.


To construct the response to the caching request $\sigma^P_t$, the algorithm $P$ runs $B$ on $\comm_t(\sigma^P_t)$.
Right after processing $\comm_t(\sigma^P_t)$, the node configuration $c$ of $B$ is well-aligned and $\sigma^P_t$ is collocated with other nodes from $S_t$ where $S_t \in \{A,B, C\}$ and $\sigma^P_t \in S_t$.
Hence, $P$ may change its cache configuration to $\textsc{cache}(c)$: such a response is feasible, because since $S_t$ is collocated, it is included by $P$ in the cache.
Furthermore, we may relate the cost of $P$ to the cost of~$B$: If $P$ modifies the cache contents, the corresponding cost is $1$, as exactly one page has to
be fetched.
Such a change occurs only if $B$ changed the configuration (at a~cost of at least $2 \cdot \alpha$).
Therefore, $2\cdot \alpha \cdot P(\sigma^P_t) \leq B(\comm_t(\sigma^P_t))$, which, summed over all requests from sequence $\sigma^P$, yields $2 \cdot
\alpha \cdot P(\sigma^P) \leq B(\sigma^B)$.

\medskip

Now we show that there exists an (offline) solution \OFF to $\sigma^B$, whose cost is exactly ${2 \cdot \alpha \cdot \OPT(\sigma^P)}$.
Recall that, for a caching request $\sigma^P_t$, $\sigma^B$ contains the~corresponding sequence $\comm_t(\sigma^P_t)$.
Before serving the first request of~$\comm_t(\sigma^P_t)$, \OFF changes its state to a well-aligned configuration corresponding to the cache of $\OPT$ right after serving caching request $\sigma^P_t$.
This ensures that the subsequence $\comm_t(\sigma^P_t)$ is free for \OFF. 

Now, we claim that the cost of node migration of \OFF is $2 \cdot \alpha$ (two caching nodes are swapped) if $\OPT$ performs a fetch, and 0 if
$\OPT$ does not change its cache contents.
Within each type of configuration with $S \in \{ B, C\}$ split, $\OPT$ may achieve a configuration with a different node from $S$ outside of the cache by exchanging the split node with any node from $S$.
To transition between any of three configuration types ((1) with $A$ (2) with $B$ split and (3) with C split), it suffices to perform a single exchange.
Therefore, $\OFF(\comm_t(\sigma^P_t)) = 2 \cdot \alpha \cdot \OPT(\sigma^P_t)$, which summed over the entire sequence $\sigma^P$ 5yields $\OFF(\sigma^B) = 2 \cdot \alpha \cdot \OPT(\sigma^P)$.

As $B$ is $\gamma$-competitive for the BRP problem, there exists a constant $\beta$, such that for any sequence $\sigma^P$ and the corresponding sequence $\sigma^B$, it holds that $B(\sigma^B) \leq \gamma \cdot \OFF(\sigma^B) +
\beta$.
Combining this inequality with the inequalities between $P$ and $B$ and between \OFF and \OPT yields
\[
	2 \cdot \alpha \cdot P(\sigma^P) \leq 
	B(\sigma^B) \leq \gamma \cdot \OFF(\sigma^B) + \beta
	 =\gamma \cdot 2 \cdot \alpha \cdot \OPT(\sigma^P) + \beta,
\]
and therefore $P$ is $\gamma$-competitive.
\end{proof}

Algorithm Alg:
maintain a counter $C_{\set{u,v}}$ for each pair of nodes $\set{u,v}$ and increment it on every remote request between $u$ and $v$.
Once $C_{\set{u,v}} = \lambda:=2\alpha$,
reset the counter and collocate the pair arbitrarily in one of the two clusters where $u$ or $v$ resides.

\begin{theorem}
	The algorithm is 4-competitive.
\end{theorem}
\begin{proof}
\mahmoud{still proof sketch}
We charge both OPT and Alg on each request that triggers a collocation.
Let $C^{t}_{\set{u,v}}$ denote the  value of $C_{\set{u,v}}$ immediately after arrival of $\sigma_t$.
Assume on request $\sigma_{t}=\set{u,v}$,
Alg collocates the pair,
which means $ C^t_{\set{u,v}} = \lambda$ immediately before Alg resets the counter.
Let $t''<t$ be the latest time (i.e.~request) when Alg collocates $\set{u,v}$ before $t$.
Let $t', t'' < t' < t$ be the  time when Alg separates $\set{u,v}$
in order to collocate some other pair sharing a node,
e.g., the pair $\set{u,v'}, v' \neq v$.
%e.g., the request $\sigma_{t'} = \set{u,v'}$.
%By Alg's definition,
%$C^{t'}_{\set{u,v'}}  = C^t_{\set{u,v}} = \lambda$.
We denote the set of all requests to $\set{u,v}$
between times $t'$ and $t$ by $\sigma_{\set{u,v}}[t',t]$.
Thus,
$| \sigma_{\set{u,v}}[t',t] | = \lambda$.
The overall cost of requests that increases the counter,
i.e.~requests in $\sigma_{\set{u,v}}[t',t]$,
and the moving cost incurred by Alg,
 is $\lambda + 2\alpha$.
%That is, the cost of all requests in $\sigma_{\set{u,v}}[t',t]$.
We distinguish three cases for the cost incurred by OPT with regards to $\set{u,v}$.
i)
OPT has $\set{u,v}$ separated through all requests in $\sigma_{\set{u,v}}[t',t]$
and therefore it pays $\lambda$.
ii)  
OPT collocates $\set{u,v}$ some time during $[t'',t]$.
Recall that whenever Alg collocates a pair,
we charge OPT only for requests up to that point (i.e., not for later requests).
Hence,
we charge the cost of  this collocation to OPT for the first time.
In particular,
we charge OPT the cost of moving one node to the cluster of the other,
e.g.~$u$ to $v$, w.l.o.g.
Therefore in this case OPT pays $\alpha$.
iii)
OPT has $\set{u,v}$ collocated before $t''$ and keeps them collocated during $[t'',t]$.
Obviously in this case OPT doesn't pays any cost for requests in $\sigma_{\set{u,v}}[t'',t]$.
However,
OPT must pay $\lambda$ for requests to $\set{u,v'}$.
By induction,
at time $t'$,
we charge OPT $\alpha$ (out of $\lambda=2\alpha$) for this pair.
Therefore,
we can charge OPT another $\alpha$ to account for requests $\sigma_{\set{u,v}}[t'',t]$.
It can be easily verified that any deviation from these cases only increases the cost to OPT.
Thus,
OPT pays at least $\alpha$ for each collocation by Alg
which gives the competitive ratio 
$(\lambda + 2\alpha)/ \alpha = 4$.

\end{proof}

\end{document}
