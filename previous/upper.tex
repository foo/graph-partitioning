
\section{An $O(kl)$-competitive Algorithm For the Scenario With the Perfect Partition}

\subsection{On a Cost of a Single Repartition}


\subsubsection{A Costly Repartition Example}

A repartition cost is bounded by $O(kl)$ (cite marcin-chen).
Now we show that the rebalance may be expensive (cubic) in terms of $k$, even for $l$ linear in $k$.

This justifies taking a different approach to repartitioning then bounding the repartition cost in $O(k^2\cdot \ell^2)$ algorithm (described in Section ``A simple upper bound'' in the DISC paper).
As the cost of a specific repartition is $\Omega(k^2)$, then the cost of DET during a single phase is: for each of $O(k \cdot l)$ edges revealed, we already have the cost $O(k^2)$ (precisely, $O(min\{k^2, k\ell\})$, and the competitive ratio would be $O(k^3\cdot \ell)$. \maciek{This would be an improvement in terms of $\ell$ still, but no hopes for $O(k\ell)$ algorithm just by improving the rebalancing analysis}

\maciek{Possibly simplify $\Omega(k^3)$? Or just present $\Omega(k^2)$?}

\subsubsection{For Constant k, Repartition Cost is Constant?}

\maciek{Interesting research problem: is a single repartition cost bounded by $O(poly(k))$? This would improve currently best-known $O(k^2\ell^2)$ to $O(k\cdot poly(k) \cdot \ell)$, i.e., would tighten the lower bound in terms of $\ell$.}
  \maciek{We have such examples for $k=3,4,5$. It gives us an $O(\ell)$-competitive algorithm for $k=3$}

\begin{lemma} \label{lemma:k=3}
  The rebalancing cost for the greedy algorithm for $k=3$ is $O(1)$.
  \maciek{Greedy = find the cheapest (closest to the current) feasible partition}
  \label{rebalancing-cost}
\end{lemma}

\begin{proof} 
% \maciek{Note: we consider a cost of cheapest feasible rebalancing after insertion of one edge. This in contrast to bounding the distance between any two reconfigurations (this theorem does not bound the latter).}
  
  We distinguish among three configuration types of algorithm's clusters: $C_1, C_2, C_3$. In $C_1$ we have $3$ singleton components (this is also the initial configuration of any cluster). In $C_2$ we have one component of size $2$ and one component of size $1$. Finally, in $C_3$ we have one component of size $3$.

  Consider a request $(u, v)$ and let $U, V$ be their clusters.
  If $U=V$, then the request does not require a reconfiguration.
  Now, we consider cases upon the type of clusters $U$ and $V$.
  Note that this is impossible that either $U$ or $V$ is $C_3$ as otherwise the resulting component would have the size $4$, and this contradicts the existence of the perfect partition.
  If either $U$ or $V$ is $C_1$, then either cluster can fit the merged component, and the rebalance is local within $U$ and $V$, for the cost of at most $3$ migrations.

  Now, we focus on the case where both $U$ and $V$ are $C_2$. Note that $(u,v)$ cannot both belong to a component of size $2$, as the merged component would have size $4$.
  If either $u$ or $v$ belongs to a component of size $2$ then it suffices to exchange components of size $1$ between $U$ and $V$.
  Finally, if $u$ and $v$ belong to components of size $1$, then we must place them in a cluster different from $U$ and $V$.
  Note that in such case, a $C_1$-type cluster exists, as otherwise the input graph would not have a perfect partition. The cost of rebalancing is then at most $12$ \maciek{can show smaller constant than 12, but it does not matter that much}.
\end{proof}

\begin{theorem}
  There exists a $O(\ell)$-competitive algorithm for $k=3$.
\mahmoud{is this needed anymore?}
\end{theorem}

\begin{proof}
  The adversary issues at most $3\ell$ requests, as otherwise the perfect partition would not exist. Each request issues at most one re-balancing for the greedy algorithm, and by Theorem~\ref{rebalancing-cost}, the total cost of any algorithm is $O(k\cdot \ell)$.
\end{proof}

We leverage the simple algorithm $ON_{k=3}$ that seeks a perfect partition for $k=3$,
towards an algorithm for the general online problem.
Note that in the general model,
an optimal partition is not a necessarily a perfect partition.
We divide the sequence of requests into phases.
A phase terminates when there is no perfect partition of components.
We apply requests to $ON_{k=3}$ as before. 
Once the current phase terminates,
we reset all components by removing (forgetting) all revealed edges,
 and then we proceed to the next phase.

\begin{corollary} \label{cor:k=3}
	There exists $O(\ell)$-competitive  algorithm for the general online problem when $k=3$. 
\end{corollary}
\begin{proof}
	Consider any phase and assume it terminates upon arrival of some request $r$.
	 By Lemma  \ref{lemma:k=3},
	ON pays $O(\ell)$ during the phase.
	Regardless of the of OPT's partition at the beginning of the phase,
	as long as OPT stays in the same configuration,
	 it incurs remote communication cost for at least one request $r'$ (possibly $r'=r$) by the end of the phase.
	 Else,
	 OPT incurs the cost of moving to new partitions.
	 In any case,
	 OPT must pay at least one during the phase.
\end{proof}
	
\subsection{Perfect Partition Learner Algorithm}


\maciek{$. \rightarrow \cdot$}
\maciek{Font for OPT, ALG, DET, dist etc}

Let $P_I = I_1, \dots, I_{\ell}$ denote the initial partition with which both ON and OPT begin and
$P_F = F_1, \dots, F_{\ell}$ the final partition.
\begin{definition}	\label{def:dist}
	The \emph{distance} of a partition $P = C_1, \dots, C_{\ell}$ is the number of nodes in $P$ that do not reside in their initial cluster.
	That is,
	$dist(P, P_I) = \sum_{j=1}^{\ell} | C_j \setminus I_j |$. 
\end{definition}

In other words,
at least $dist(P, P_I)/2$ node swaps are required in order to reach the partition $P$ from $P_I$.
Therefore,
$OPT \geq \Delta:= dist(P_F, P_I) $.
We devise an online algorithm that mimics OPT by minimizing the distance to the initial partition $P_I$ with each re-partitioning.
As a result,
ON never ends up in a partition that is further away from $P_I$ than $\Delta$.
This invariant ensures that ON does not pay too much while reaching $P_F$.

\begin{algorithm}
	\renewcommand{\algorithmicrequire}{\textbf{Input:}}
	\renewcommand{\algorithmicensure}{\textbf{Output:}}
	\begin{algorithmic}[1]
		\Require 
		$k, \ell$,
		initial partition $P_I$,
		sequence of  edges $\sigma_1, \dots, \sigma_N$ 
		\Ensure Final partition $P_F$ 
		\State for each node $v$ create a singleton component $C_v$ and add it to $\mathcal{C}$ \label{line:initcomponents}
		\State on communication request $\sigma_t=\{u,v\}$:
		\State Let $C_1 \ni u$ and $C_2 \ni v$ be the container components
		\If{$C_1 \neq C_2$}
		\State unite the two components into a single component $C'$ and
		$\mathcal{C} = (\mathcal{C}\setminus\set{C_1, C_2}) \cup ~\set{C'}$ \label{line:mergecomponents}
		\If{cluster$(C_1) \neq$ cluster$(C_2)$}		
		\State \textit{re-partition($k, \ell, P_I, \mathcal{C}$)} \label{line:rebalance} 
		\EndIf
		\EndIf
	\end{algorithmic}
	\caption{Perfect Partition Learner}
	\label{alg:ppl}
      \end{algorithm}

      \maciek{``re-partition'' procedure name should indicate the fact that this is a specific repartition that is close to initial configuration}

\mahmoud{address the \textit{re-partition()}}

\begin{property} \label{prop:dist<OPT}
	Let $P$ be any partition chosen by Algorithm \ref{alg:ppl} at Line $\ref{line:rebalance}$.
	Then, $dist(P,P_I) \leq \Delta$.
\end{property}

\begin{lemma}	\label{lemma:rebalancecost}
	The cost of re-partitioning at Line \ref{line:rebalance} is at most $2.OPT$.
\end{lemma}
\begin{IEEEproof}
	Consider the re-partitioning that transforms $P_{t-1}$ to $P_t$ upon the request $\sigma_t$.
	Let $M \subset V$ denote the set of nodes that are migrated during this process.
	We have $M = M^+ \cup M^- \cup M^\circ$,
	where $M^-$ ($M^+$) denotes the subset of nodes that
	enter (leave) their original cluster during the re-partitioning,
	and $M^\circ$ denotes the set of remaining nodes in $M$.
	Since $|M^- \cup M^\circ|$ nodes are not in their original cluster before the re-partitioning,
	by Definition \ref{def:dist},
	the distance before the re-partitioning is $dist(P_{t-1}) \geq | M^- \cup M^\circ |$.
	Analogously,
	 the distance afterwards is $dist(P_{t}) \geq | M^+ \cup M^\circ |$.
	Thus,
	$|M| \leq dist(P_{t-1}) + dist(P_{t})$.
	By Property \ref{prop:dist<OPT},
	$dist(P_{t-1}) , dist(P_{t}) \leq \Delta \leq OPT$
	and thereby we have	
	$|M| \leq 2.OPT$.
\end{IEEEproof}

\begin{theorem}	\label{thm:upperbound}
	Algorithm \ref{alg:ppl} reaches the final partition while being $(2.k.\ell)$-competitive.
\end{theorem}
\begin{IEEEproof}
	The algorithm eventually reaches the final partition since it
	 evaluates all possible $\ell$-way partitions of components on each external request,
	including the request that completes revealing of all ground truth components.
	There are at most $(k-1).\ell < k.\ell $ calls to \emph{re-partition()} and by Lemma \ref{lemma:rebalancecost} each costs at most $2.OPT$.
	The total cost is therefore at most $2.OPT.k.\ell$ which implies the competitive ratio.
      \end{IEEEproof}

      \section{$O(k)$-competitive algorithm for the ring}

      \maciek{todo}

      We know that the lower bound is $k$ for the ring communication pattern (cite Marcin-Chen).
      The simple matching (assymptotically \maciek{maybe it is easy to have $k$ exactly?}) upper bound is to fix the reconfiguration to be the cyclic shift. We reconfigure upon encountering inter-cluster request. We use phases: upon discovering the edge on each cut we charge OPT 1 and reset counters.